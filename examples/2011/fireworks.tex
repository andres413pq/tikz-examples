% Fireworks
% Author: percusse
\documentclass{article}
\usepackage{tikz}
%%%<
\usepackage{verbatim}
\usepackage[active,tightpage]{preview}
\PreviewEnvironment{tikzpicture}
\setlength\PreviewBorder{0pt}%
%%%>
\begin{comment}
:Title: Fireworks
:Tags: Fadings, Scopes, Transparency,
:Author: percusse
:Slug: fireworks
Posted for celebrating Christmas 2011 on http://tex.stackexchange.com/q/39485/213

Displayed with TeXworks PDF previewer. Some PDF viewers don't display
correctly, noticed with Adobe Reader on Linux and Evince on Linux.
\end{comment}
\usetikzlibrary{calc,decorations.pathmorphing}
%
\pgfdeclareradialshading{someshade}{\pgfpointorigin}{%
  color(0mm)=(pgftransparent!40);color(5mm)=(pgftransparent!50);%
  color(10mm)=(pgftransparent!70);color(2cm)=(pgftransparent!100)}
\pgfdeclareradialshading{somenodeshade}{\pgfpointorigin}{%
  color(0mm)=(pgftransparent!0);color(2mm)=(pgftransparent!5);%
  color(5mm)=(pgftransparent!95);color(20mm)=(pgftransparent!100)}
\pgfdeclareradialshading{invertshade}{\pgfpointorigin}{%
  color(0mm)=(pgftransparent!100);color(6mm)=(pgftransparent!95);%
  color(10mm)=(pgftransparent!60);color(2cm)=(pgftransparent!0)}
\pgfdeclarefading{fadeit}{\pgfuseshading{someshade}}
\pgfdeclarefading{fadein}{\pgfuseshading{invertshade}}
%
\begin{document}
\begin{tikzpicture}[projectile/.style={decorate,decoration={random steps,
  segment length=3pt,amplitude=0.5pt}}]
  \fill[black] (-4,-4) rectangle (6,5);

  \begin{scope}[xshift=0cm,yshift=-0.4cm,transparency group]
    \pgfsetfading{fadein}{\pgftransformshift{\pgfpointorigin}}
    \foreach \x in {0,6,..., 360}{\draw[blue!80!white,projectile,line width=1.1pt]
      (0,0) to [in=90] (10*rand+\x:rand*1mm+2cm);};
  \end{scope}

  \begin{scope}[xshift=2cm,yshift=1cm]
    \foreach \x in {0,8,..., 360}{\draw [yellow!5,thick,projectile] (0.7,0)
      to  (3*rand+\x :1mm*rand+2.2cm)  node[circle,inner sep=1mm,
      shade,shading=somenodeshade,opacity=0.1] {};}
    {\pgfsetfading{fadeit}{\pgftransformshift{\pgfpoint{2.5cm}{1cm}}}};
    \fill[white] (-3,-3) rectangle (3,3);
  \end{scope}

  \begin{scope}[xshift=3cm,yshift=-1cm]
    \foreach \x in {0,10,..., 360}{\def\r1{rand}\draw [yellow]
      ($(0,0)!abs{\r1}!(\x :5mm)$) to [in=90] ($(0,0)!abs{\r1}+0.2!(\x :8mm)$);}
    {\pgfsetfading{fadeit}{\pgftransformshift{\pgfpoint{3cm}{-1cm}}}};
    \fill[yellow,opacity=0.6] (-3,-3) rectangle (3,3);
  \end{scope}

  \begin{scope}[xshift=-1cm,yshift=1.5cm]
    \foreach \x in {0,12,..., 360}{\def\r2{rand}\draw [red,line width=0.5pt]
      ($(0,0)!abs{\r2}!(\x :3mm)$) -- ($(0,0)!abs{\r2}+0.1!(\x :7mm)$);}
    {\pgfsetfading{fadeit}{\pgftransformshift{\pgfpoint{-1cm}{1.5cm}}}};
    \fill[red,opacity=0.6] (-3,-3) rectangle (3,3);
  \end{scope}
\end{tikzpicture}
\end{document}
