% Author: Julien Cretel
% Date:   24/02/2013
\documentclass{article}
\usepackage{tikz}
%%%<
\usepackage{verbatim}
\usepackage[active,tightpage]{preview}
\PreviewEnvironment{center}
\setlength\PreviewBorder{20pt}%
%%%>
\begin{comment}
:Title: Escher Brick and Penrose Triangle
:Tags: Styles;Coordinate Calculations;Geometry;Decorative Drawings
:Author: Julien Cretel
:Slug: escher-brick-penrose-triangle

The first picture draws an impossible brick, which induces
an optical illusion similar to that triggered by Escher's
impossible cube.

The second picture draws a Penrose triangle, another similar
optical illusion.
\end{comment}	
\begin{document}
  \begin{center}
    \begin{tikzpicture}[scale=4.5, line join=bevel]
	
      % \a and \b are two macros defining characteristic
      % dimensions of the impossible brick.
      \pgfmathsetmacro{\a}{0.18}
      \pgfmathsetmacro{\b}{1.37}

      \tikzset{
        brick_edges/.style  = {thick,black},
        face_colourA/.style = {fill=gray!50},
        face_colourB/.style = {fill=gray!25},
        face_colourC/.style = {fill=gray!90},
      }
		
      % Faces
      \fill[face_colourA]
        (0,0)             --
        ++(\b,0)          --
        ++(-\a,-\a)       --
        ++({-\b+2*\a},0)  --
        ++(0,-{2*\a})     --
        ++(\b,0)          --
        ++(-\a,-\a)       --
        ++(-\b,0);        --
        cycle;
      \fill[face_colourB]
        (0,0)             --
        ++(\a,\a)         --
        ++(\b,0)          --
        ++(0,-{4*\a})     --
        ++(-\b,0)         --
        ++(\a,\a)         --
        ++({\b-2*\a},0)   --
        ++(0,{2*\a})      --
        cycle;
      \fill[face_colourC]
        (\b,0)            --
        ++(0,{-2*\a})     --
        ++(-\a,0)         --
        ++(0,\a)          --
        cycle;
      \fill[face_colourC]
        (\a,-\a)          --
        ++(\a,0)          --
        ++(0,-\a)         --
        ++(-\a,-\a)       --
        cycle;
      % edges
      \draw[brick_edges]
        (0,0)             --
        ++(\a,\a)         --
        ++(\b,0)          --
        ++(0,-{4*\a})     --
        ++(-\b,0)         --
        ++(0,{2*\a})      --
        ++({\b-2*\a},0)   --
        ++(0,-\a);
      \draw[brick_edges]
        ({\b+\a},{-3*\a}) --
        ++(-\a,-\a)       --
        ++(-\b,0)         --
        ++(0,{4*\a})      --
        ++(\b,0)          --
        ++(0,-{2*\a})     --
        ++({-\b+2*\a},0)  --
        ++(0,\a);
      \draw[brick_edges]
        (\b,0) -- ++(-\a,-\a);
      \draw[brick_edges]
        (\a,{-3*\a}) -- ++(\a,\a);
    \end{tikzpicture}

    \vspace{1cm}

	\begin{tikzpicture}[scale=1, line join=bevel]
	
    % \a and \b are two macros defining characteristic
    % dimensions of the Penrose triangle.		
    \pgfmathsetmacro{\a}{2.5}
    \pgfmathsetmacro{\b}{0.9}

    \tikzset{%
      apply style/.code     = {\tikzset{#1}},
      triangle_edges/.style = {thick,draw=black}
    }

    \foreach \theta/\facestyle in {%
        0/{triangle_edges, fill = gray!50},
      120/{triangle_edges, fill = gray!25},
      240/{triangle_edges, fill = gray!90}%
    }{
      \begin{scope}[rotate=\theta]
        \draw[apply style/.expand once=\facestyle]
          ({-sqrt(3)/2*\a},{-0.5*\a})                     --
          ++(-\b,0)                                       --
            ({0.5*\b},{\a+3*sqrt(3)/2*\b})                -- % higher point	
            ({sqrt(3)/2*\a+2.5*\b},{-.5*\a-sqrt(3)/2*\b}) -- % rightmost point
          ++({-.5*\b},-{sqrt(3)/2*\b})                    -- % lower point
            ({0.5*\b},{\a+sqrt(3)/2*\b})                  --
          cycle;
        \end{scope}
      }	
	  \end{tikzpicture}
  \end{center}
\end{document}