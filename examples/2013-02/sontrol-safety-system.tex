% Principle of separated control and safety systems.
% Author: Erno Pentzin (2013)
\documentclass{article}
\usepackage{tikz} 
%%%<
\usepackage{verbatim}
\usepackage[active,tightpage]{preview}
\PreviewEnvironment{tikzpicture}
\setlength\PreviewBorder{10pt}%
%%%>
\begin{comment}
:Title: Principle of separated control and safety systems
:Tags: Decorations;Markings
:Author: Erno Pentzin
:Slug: sontrol-safety-system

Adapted from the figure in the book by Gruhn and Cheddie (2006).
\end{comment}
\usetikzlibrary{decorations.markings}
\begin{document}

\begin{tikzpicture}[
	decoration={
		markings,
		mark=at position 1cm with {\arrow[black]{stealth}},
	},
	path/.style={
		->,
		>=stealth,
		postaction=decorate
	},
	every node/.style={font=\sffamily}
]

% Boxes
\node [
	draw,
	yshift=2cm,
	minimum width=4cm,
] (control) {Control System};
\node [
	draw,
	minimum width=5cm,
	minimum height=1cm,
] (process) {Process/Equipment};
\node [
	draw,
	yshift=-2cm,
	minimum width=4cm,
] (safety) {Safety System};

% Connecting Arrows
\draw[path] (control.east) -- ++(1,0) |- ([yshift=-0.25cm] process.north east);
\draw[path] ([yshift=-0.25cm] process.north west) -- ++(-1,0) |- (control.west);
\draw[path] ([yshift=0.25cm] process.south west) -- ++(-1,0) |- (safety.west);
\draw[path] (safety.east) -- ++(1,0) |- ([yshift=0.25cm] process.south east);

\end{tikzpicture}

\end{document}
