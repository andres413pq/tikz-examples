\documentclass[x11names]{article}
\usepackage[width=16cm]{geometry}
\usepackage{tikz}
\usetikzlibrary{calc,positioning,shadows,decorations,decorations.pathmorphing,
   hobby,shapes.geometric}

% original code by percusse:
% http://tex.stackexchange.com/questions/39296/simulating-hand-drawn-lines#49961
\makeatletter
\pgfdeclaredecoration{penciline}{initial}{
    \state{initial}[width=+\pgfdecoratedinputsegmentremainingdistance,auto corner on length=3mm,]{
        \pgfpathcurveto%
        {% From
            \pgfqpoint{\pgfdecoratedinputsegmentremainingdistance}
                            {\pgfdecorationsegmentamplitude}
        }
        {%  Control 1
        \pgfmathrand
        \pgfpointadd{\pgfqpoint{\pgfdecoratedinputsegmentremainingdistance}{0pt}}
                        {\pgfqpoint{-\pgfdecorationsegmentaspect\pgfdecoratedinputsegmentremainingdistance}%
                                        {\pgfmathresult\pgfdecorationsegmentamplitude}
                        }
        }
        {%TO 
        \pgfpointadd{\pgfpointdecoratedinputsegmentlast}{\pgfpoint{1pt}{1pt}}
        }
    }
    \state{final}{}
}
\makeatother

\tikzset{candle decoration/.style={decorate, decoration={random steps,segment length=2pt,amplitude=0.3pt}}}
\tikzset{candle shadow/.style={drop shadow={shadow xshift=.4ex,shadow yshift=-.3ex}}}
\tikzset{stick candle/.style={
            draw,decorate, decoration={penciline,amplitude=3pt},rectangle, 
            anchor=north, minimum width=0.5cm, minimum height=#1,
            left color=white, right color=Honeydew2!80
            },
            stick candle/.default={2cm}
}

\tikzset{candle support style/.style={
            draw,rectangle, 
            decorate, decoration={penciline,amplitude=1.5pt},
            anchor=north, minimum width=0.9cm, minimum height=0.2cm,
            left color=Goldenrod1!40, right color=Goldenrod2,
            candle shadow
            },
}

\tikzset{candle base style/.style={
            draw,semicircle,rotate=183,
            candle decoration,
            minimum width=0.6cm, minimum height=0.2cm,
            left color=Goldenrod1!40, right color=Goldenrod2,
            candle shadow
            },
}

\newcounter{candles}
\setcounter{candles}{1}
\tikzset{candle/.code={
        \draw[candle decoration, left color=Gold1!20, right color=Goldenrod1, candle shadow]
        (0,1) to[curve through={(-0.1,0.5)..(-0.2,0) .. (0.2,0) .. (0.1,0.5)}] (0,1);       
        \draw[draw=none,candle decoration, fill=LemonChiffon1]
        (0,0.45) to[curve through={(-0.05,0.25)..(-0.1,0) .. (0.1,0) .. (0.05,0.25)}] (0,0.45);
        \draw[decorate, decoration=penciline](0,0.125)--(0,-0.25)node(candle\thecandles){};     
        \node[stick candle=#1, candle shadow] (candlesupport\thecandles) at (candle\thecandles){}; 
        \node[candle support style, below=-0.1cm of candlesupport\thecandles](basecandle\thecandles) {};
        \node[candle base style, below=0.275cm of basecandle\thecandles](downbasecandle\thecandles){};
        \stepcounter{candles}
    }
}

\tikzset{candelabrum style/.style={
            anchor=north,draw,trapezium, trapezium stretches=true, 
            candle decoration,
            minimum height=5cm, minimum width=0.9cm,
            left color=Goldenrod1!40, right color=Goldenrod2,
            candle shadow
            },
}

% original code by Paul Gaborit:
% tex.stackexchange.com/questions/72784/arrow-with-two-colors-with-tikz/#72793
\tikzset{
  double path/.style args={#1 colored by #2 and #3}{
    -,line join=bevel,line cap=rect,
    shorten >=0.04cm,
    shorten <=0.04cm,
    line width=#1,#2, % first path
    postaction={draw,-,#3,line width=(#1)/1.5,
                shorten <=(#1)/4,shorten >=2*(#1)/4}, % second path
  }
}

\tikzset{candelabrum branch/.style={
        double path=3pt colored by black!80!Goldenrod1 and Goldenrod1!60,bend #1,
        candle decoration,
    }
}

\begin{document}
\begin{tikzpicture}[remember picture]
\foreach \xpos in {0,1.2,2.4,3.6}{
\begin{scope}[xshift=\xpos cm,yshift=-1cm]
\node[candle=1cm]{};
\end{scope}
}
\end{tikzpicture}
\begin{tikzpicture}[remember picture]
\node[candle]{};
\end{tikzpicture}
\begin{tikzpicture}[remember picture]
\foreach \xpos in {0,1.2,2.4,3.6}{
\begin{scope}[xshift=\xpos cm,yshift=-1cm]
\node[candle=1cm]{};
\end{scope}
}
\end{tikzpicture}
% Candelabrum
\begin{tikzpicture}[remember picture, overlay]
\node[candelabrum style] (candelabrum) at (downbasecandle5.north){};
%\node[candelabrum base] at (candelabrum.south){};
\draw[left color=Goldenrod1!40, right color=Goldenrod2,candle decoration, candle shadow]
($(candelabrum.bottom left corner)-(0.4,0.6)$) parabola[bend at end] (candelabrum.bottom left corner)--
(candelabrum.bottom right corner) parabola ($(candelabrum.bottom right corner)+(0.4,-0.6)$)
--($(candelabrum.bottom left corner)-(0.4,0.6)$);
\draw[left color=Goldenrod1!40, right color=Goldenrod2,candle decoration, candle shadow]($(candelabrum.bottom left corner)-(0.4,0.6)$)-- ($(candelabrum.bottom right corner)+(0.4,-0.6)$)-- ($(candelabrum.bottom right corner)+(0.4,-0.9)$)--($(candelabrum.bottom left corner)-(0.4,0.9)$)--cycle;
% left
\path (downbasecandle1.north)edge[candelabrum branch=right](candelabrum.195);
\path (downbasecandle2.north)edge[candelabrum branch=right](candelabrum.115);
\path (downbasecandle3.north)edge[candelabrum branch=right](candelabrum.100);
\path (downbasecandle4.north)edge[candelabrum branch=right](candelabrum.97);
% right
\path (downbasecandle6.north)edge[candelabrum branch=left](candelabrum.83);
\path (downbasecandle7.north)edge[candelabrum branch=left](candelabrum.80);
\path (downbasecandle8.north)edge[candelabrum branch=left](candelabrum.65);
\path (downbasecandle9.north)edge[candelabrum branch=left](candelabrum.345);
\end{tikzpicture}
\end{document}

