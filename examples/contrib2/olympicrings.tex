% Olympic rings
% Author: Paul Gaborit
\documentclass[a4paper]{article}
\usepackage{tikz}
%%%<
\usepackage{verbatim}
\usepackage[active,floats,tightpage]{preview}
\PreviewEnvironment{tikzpicture}
\setlength\PreviewBorder{5pt}%
%%%>

\begin{document}

\begin{comment}
:Title: The Olympic rings

\end{comment}


\begin{tikzpicture}
 \definecolor{r1}{RGB}{0,129,188}
 \definecolor{r2}{RGB}{252,177,49}
 \definecolor{r3}{RGB}{35,34,35}
 \definecolor{r4}{RGB}{0,157,87}
 \definecolor{r5}{RGB}{238,50,78}
 \begin{scope}
   \clip (-6,2) rectangle (6,-.9);
   \foreach \col/\xp/\yp in {
     r5/4/0, r4/2/-1.8, r3/0/0,
     r2/-2/-1.8, r1/-4/0
   } {
     \path[draw=white,line width=.08cm,
     fill=\col,even odd rule]
     (\xp, \yp) circle (1.9cm)
     (\xp, \yp) circle (1.5cm);
   }
 \end{scope}
 \begin{scope}
   \clip (-6,-.9) rectangle (6,-3.8);
   \foreach \col/\xp/\yp in {
     r1/-4/0, r2/-2/-1.8, r3/0/0,
     r4/2/-1.8, r5/4/0
   } {
     \path[draw=white,line width=.08cm,
     fill=\col,even odd rule]
     (\xp, \yp) circle (1.9cm)
     (\xp, \yp) circle (1.5cm);
   }
 \end{scope}
\end{tikzpicture}
\end{document}