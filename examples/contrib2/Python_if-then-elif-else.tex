% Author: Dr. Ludger Humbert
% Source: https://haspe.homeip.net/projekte/ddi/browser/tex/pgf2                                                                             
% https://haspe.homeip.net/projekte/ddi/browser/tex/pgf2/Python_if-then-elif-else.tex $:
%

\documentclass{article}
\usepackage[ngerman]{babel}

\usepackage{tikz}
%%%<
\usepackage{verbatim}
\usepackage[active,tightpage]{preview}
\PreviewEnvironment{tikzpicture}
\setlength\PreviewBorder{10pt}%
%%%>

\begin{comment}
:Title: Python if-then-else syntax diagram

A `syntax diagram`_ (or railroad diagram) for a if-then-elif-else expression in Python. 

.. _syntax diagram: http://en.wikipedia.org/wiki/Syntax_diagram

:Author: Dr. `Ludger Humbert`_
:Source: `https://haspe.homeip.net/projekte/ddi/browser/tex/pgf2`__

.. __: https://haspe.homeip.net/projekte/ddi/browser/tex/pgf2
.. _Ludger Humbert: https://haspe.homeip.net/cgi-bin/pyblosxom.cgi

\end{comment}
\usepackage{xxcolor}


\usetikzlibrary{chains,matrix,scopes,decorations.shapes,arrows,shapes}

% fuer Railroad-Diagramme
\tikzset{
  nonterminal/.style={
    % The shape:
    rectangle,
    % The size:
    minimum size=6mm,
    % The border:
    very thick,
    draw=red!50!black!50,         % 50% red and 50% black,
                                  % and that mixed with 50% white
    % The filling:
    top color=white,              % a shading that is white at the top...
    bottom color=red!50!black!20, % and something else at the bottom
    % Font
    font=\itshape
  },
  terminal/.style={
    % The shape:
    rounded rectangle,
    minimum size=6mm,
    % The rest
    very thick,draw=black!50,
    top color=white,bottom color=black!20,
    font=\ttfamily},
  skip loop/.style={to path={-- ++(0,#1) -| (\tikztotarget)}}
}
{
  \tikzset{terminal/.append style={text height=1.5ex,text depth=.25ex}}
  \tikzset{nonterminal/.append style={text height=1.5ex,text depth=.25ex}}
}


\renewcommand\familydefault{\sfdefault}

\listfiles

\begin{document}


%% if_stmt ::= "if" expression ":" suite
%%                ( "elif" expression ":" suite )*
%%                ["else" ":" suite]
%%
\hspace{-35ex}
\begin{tikzpicture}[point/.style={coordinate},>=stealth',thick,draw=black!50,
                    tip/.style={->,shorten >=0.007pt},every join/.style={rounded corners},
                    hv path/.style={to path={-| (\tikztotarget)}},
                    vh path/.style={to path={|- (\tikztotarget)}},
                    text height=1.5ex,text depth=.25ex]     % um die Hoehe des Punktes festzuzurren
  \matrix[ampersand replacement=\&,column sep=4mm] {
    \node (p1)  [point]  {}; \&    \node (ifs)    [terminal]     {if};             \&
    \node (p2)  [point]  {}; \&    \node (expr1)  [nonterminal]  {expr};           \&
    \node (p3)  [point]  {}; \&    \node (colon1) [terminal]     {:};              \&
    \node (p4)  [point]  {}; \&    \node (suite1) [nonterminal]  {suite};          \&
    \node (p5)  [point]  {}; \&    \node (p6)     [point]  {};                     \&
    \node (p6a) [point]  {}; \&    \node (p6b)    [point]  {};                     \&
    \node (p7)  [point]  {}; \&    \node (elif)   [terminal]     {elif};           \&
    \node (p8)  [point]  {}; \&    \node (expr2)  [nonterminal]  {expr};           \&
    \node (p9)  [point]  {}; \&    \node (colon2) [terminal]     {:};              \&
    \node (p10) [point]  {}; \&    \node (suite2) [nonterminal]  {suite};          \&
    \node (p11) [point]  {}; \&    \node (p12)    [point]  {};                     \&
    \node (p12a)[point]  {}; \&    \node (p12b)   [point]  {};                     \&
    \node (p13) [point]  {}; \&    \node (else)   [terminal]     {else};           \&
    \node (p14) [point]  {}; \&    \node (colon3) [terminal]     {:};              \&
    \node (p15) [point]  {}; \&    \node (suite3) [nonterminal]  {suite};          \&
    \node (p16) [point]  {}; \&    \node (p17)    [point]       {};\\
  };

  { [start chain]
    \chainin (p1);
    \chainin (ifs)    [join=by tip];
    \chainin (p2)     [join];
    \chainin (expr1)  [join=by tip];
    \chainin (p3)     [join];
    \chainin (colon1) [join=by tip];
    \chainin (p4)     [join];
    \chainin (suite1) [join=by tip];
    \chainin (p7)     [join];
    \chainin (elif)   [join=by tip];
    \chainin (p8)     [join];
    \chainin (expr2)  [join=by tip];
    \chainin (p9)     [join];
    \chainin (colon2) [join=by tip];
    \chainin (p10)    [join];
    \chainin (suite2) [join=by tip];
    \chainin (p12a)   [join,join=with p6 by {skip loop=-11mm,tip}];
    \chainin (p13)    [join];
    \chainin (else)   [join=by tip];
    \chainin (p14)    [join];
    \chainin (colon3) [join=by tip];
    \chainin (p15)    [join];
    \chainin (suite3) [join=by tip];
    \chainin (p16)    [join];
    \chainin (p17)    [join=by tip];
    \chainin (p16)    [join,join=with p12b by {skip loop=-11mm,tip}];
    \chainin (p6b)    [join=with p11 by {skip loop=11mm,tip}];
  }
\end{tikzpicture}



\end{document}

