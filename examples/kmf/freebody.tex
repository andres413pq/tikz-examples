\documentclass{article}
\usepackage{tikz}
\usetikzlibrary{scopes}

%%%<
\usepackage{verbatim}
\usepackage[active,tightpage]{preview}
\PreviewEnvironment{tikzpicture}
\setlength\PreviewBorder{5pt}%
%%%>

\begin{comment}

:Title: Free body diagrams
:Slug: free-body-diagrams
:Tags: Matrices

Illustration of the classical "bodies connected by a string" problem.
A great advantage of using TikZ for drawing illustrations like this, is that the
drawings can be parameterized. In this example the inclination angle is
parameterized. PGF's mathematical engine is then used to do the necessary calculations. 

Note that this example uses the shorthand scope notation in some places. 

\end{comment}

\begin{document}

\def\iangle{35} % Angle of the inclined plane
\def\down{-90}
\def\arcr{0.5cm} % Radius of the arc used to indicate angles

\begin{tikzpicture}[
    force/.style={>=latex,draw=blue,fill=blue},
    axis/.style={densely dashed,gray,font=\small},
    M/.style={rectangle,draw,fill=lightgray,minimum size=0.5cm,thin},
    m/.style={rectangle,draw=black,fill=lightgray,minimum size=0.3cm,thin},
    plane/.style={draw=black,fill=blue!10},
    string/.style={draw=red, thick},
    pulley/.style={thick},
]
\matrix[column sep=1cm] {
    %% Sketch
    \draw[plane] (0,-1) coordinate (base)
                     -- coordinate[pos=0.5] (mid) ++(\iangle:3) coordinate (top)
                     |- (base) -- cycle;
    \path (mid) node[M,rotate=\iangle,yshift=0.25cm] (M) {};
    \draw[pulley] (top) -- ++(\iangle:0.25) circle (0.25cm)
                   ++ (90-\iangle:0.5) coordinate (pulley);
    \draw[string] (M.east) -- ++(\iangle:1.5cm) arc (90+\iangle:0:0.25)
                  -- ++(0,-1) node[m] {};

    \draw[->] (base)++(\arcr,0) arc (0:\iangle:\arcr);
    \path (base)++(\iangle*0.5:\arcr+5pt) node {$\alpha$};
    %%
&
    %% Free body diagram of M
    \begin{scope}[rotate=\iangle]
        \node[M,transform shape] (M) {};
        % Draw axes and help lines
        {[axis,->]
            \draw (0,-1) -- (0,2) node[right] {$+y$};
            \draw (M) -- ++(2,0) node[right] {$+x$};
            % Indicate angle. The code is a bit awkward.
            \draw[solid,shorten >=0.5pt] (\down-\iangle:\arcr)
                arc(\down-\iangle:\down:\arcr);
            \node at (\down-0.5*\iangle:1.3*\arcr) {$\alpha$};
        }
        % Forces
        {[force,->]
            % Assuming that Mg = 1. The normal force will therefore be cos(alpha)
            \draw (M.center) -- ++(0,{cos(\iangle)}) node[above right] {$N$};
            \draw (M.west) -- ++(-1,0) node[left] {$f_R$};
            \draw (M.east) -- ++(1,0) node[above] {$T$};
        }
    \end{scope}
    % Draw gravity force. The code is put outside the rotated
    % scope for simplicity. No need to do any angle calculations. 
    \draw[force,->] (M.center) -- ++(0,-1) node[below] {$Mg$};
    %%
&
    %%%
    % Free body diagram of m
    \node[m] (m) {};
    \draw[axis,->] (m) -- ++(0,-2) node[left] {$+$};
    {[force,->]
        \draw (m.north) -- ++(0,1) node[above] {$T'$};
        \draw (m.south) -- ++(0,-1) node[right] {$mg$};
    }
\\
};
\end{tikzpicture}

\end{document}
