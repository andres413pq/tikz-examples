% Regular polygons
\documentclass{article}
\usepackage{tikz}
%%%<
\usepackage{verbatim}
\usepackage[active,tightpage]{preview}
\PreviewEnvironment{tikzpicture}
\setlength\PreviewBorder{10pt}%
%%%>

\begin{comment}
:Title: Regular polygons
:Tags: Paths, Foreach, PGF 1.18

Example of how to draw `regular polygons`_ using a ``\foreach`` loop inside a path.
The angles have been calculated manually. 

.. admonition:: Update. PGF 1.18

   - Fractional angular values are now supported 
   - Regular polygons are now available as node shapes! See the `node shapes example`_ 
     or the manual for more details.


:Source: The `pgf-help forum`_

.. _regular polygons: http://mathworld.wolfram.com/RegularPolygon.html
.. _pgf-help forum: http://sourceforge.net/forum/message.php?msg_id=4284541
.. _node shapes example: /pgftikzexamples/node-shapes/

\end{comment}

\begin{document}

% Radius of regular polygons
\newdimen\R
\R=0.8cm

\begin{tikzpicture}
    % Indicate the boundary of the regular polygons
    \draw [thin,black!20] circle (\R) ;
    \fill[black!20] circle (2pt);
    \draw (0:\R) \foreach \x in {120,240} {
            -- (\x:\R)
        } -- cycle (90:\R) node[above] {$n=3$} ;
    \draw[xshift=2.5\R] (0:\R) \foreach \x in {90,180,...,359} {
            -- (\x:\R)
        } -- cycle (90:\R) node[above] {$n=4$} ;
    \draw[xshift=5.0\R] (0:\R) \foreach \x in {72,144,...,359} {
            -- (\x:\R)
        } -- cycle (90:\R) node[above] {$n=5$} ;
    \begin{scope}[yshift=-3\R]
        \draw (0:\R) \foreach \x in {60,120,...,359} {
                -- (\x:\R)
            }-- cycle (90:\R) node[above] {$n=6$} ;
            
        % 360/7 = 51.4286 For PGF v < 1.18 we have to round to the nearest
        % integer. Newer version support fractional angle values.
        % For a more accurate result use the sequence
        % {51, 103, 154, 206, 257, 309}
        %
        \draw[xshift=2.5\R] (0:\R) \foreach \x in {51.4286,102.8571,...,359} {
                -- (\x:\R)
            }-- cycle (90:\R) node[above] {$n=7$} ;
        \draw[xshift=5.0\R] (0:\R) \foreach \x in {45,90,...,359} {
                -- (\x:\R)
            } -- cycle (90:\R) node[above] {$n=8$} ;
    \end{scope}
    \draw[yshift=-6.0\R] (0:\R) \foreach \x in {10,20,...,359} {
            -- (\x:\R)
        } -- cycle (90:\R) node[above] {$n=36$} ;
\end{tikzpicture}

\end{document}