% Animated set intersection
%
% The animations can only be viewed in Acrobat (Reader) v >= 6.
\documentclass{beamer}
\usepackage{animate}
%%%<
\usepackage{verbatim}
%%%>

\begin{comment}
:Title: Animated set intersection
:Tags: Beamer, Foreach


The `animate.sty`_ package is a useful and fun tool for creating Javascript driven animations in
PDF files. Brian S. Marks has already contributed an impressive `animated statistical distribution`_ example that demonstrates `animate.sty`_. The following example is much simpler, but demonstrates a more general technique that is useful in conjunction with other tools as well.

.. _animated statistical distribution: http://www.fauskes.net/pgftikzexamples/animated-distributions/

The animation is a two step process:

1. Creating the individual frames of the animation. Each frame becomes a page in a PDF file.
2. Putting together the frames using animate.

The main reason we do it this way is because animate does not allow a ``\newframe`` *inside* PGF's ``\foreach`` command. This means that we can't put a ``\foreach`` loop  *around* the parameterized picture inside the ``animateinline`` environment (we can use a ``\foreach`` for other purposes though).

**Update**: `A standalone version`_ of the example is now available that uses ``\whiledo`` to parameterize the picture. 

Step 1. Creating the frames
---------------------------

The animation we will make demonstrates the set intersection operation. The code for drawing the sets is based on the `venn diagram`_ example. A ``foreach`` loop is used to draw a total of 21 frames, varying the x-coordinate of set B. The preview.sty_ package is used to crop every tikzpicture and put each picture/frame on a separate page.

.. _venn diagram: http://www.fauskes.net/pgftikzexamples/venn-diagram/
.. _preview.sty: http://www.ctan.org/tex-archive/help/Catalogue/entries/preview.html

`setanim.tex`_ (setanim.pdf_) source:

.. _setanim.tex: animations/setanim.tex
.. _setanim.pdf: animations/setanim.pdf

.. sourcecode:: latex

    % Animate a set intersection operation.
    % The target document is a beamer presentation, so we load the
    % beamer class to get the same fonts as in the presentation.
    \documentclass{beamer}
    \usepackage{tikz}

    \usepackage[active,tightpage]{preview}
    \PreviewEnvironment{tikzpicture}

    \begin{document}
    % define the bounding box
    \def\boundb{(-2,2) rectangle (4,-2)}
    % Create n frames with \xb as parameter
    \foreach \xb in {0,0.1,...,2.1}{
        \def\setA{(0,0) circle (1)}
        \def\setB{(\xb,0) circle (1)}

        \begin{tikzpicture}
            \draw \boundb;
            % intersection
            \begin{scope}
                \clip \setA;
                \fill[black!20] \setB;
            \end{scope}
            \begin{scope}[even odd rule]% first circle without the second
                \clip \setB \boundb;
                \fill[red!20] \setA;
            \end{scope}
            \begin{scope}[even odd rule]% first circle without the second
                \clip \setA \boundb;
                \fill[blue!20] \setB;
            \end{scope}
            \draw \setA;
            \draw \setB;
            \node at (-1,0) [left] {$A$};
            \node at (\xb+1,0) [right] {$B$};
            \node at (4,2) [below left] {$A\cap B$};
        \end{tikzpicture}
    }
    \end{document}


The output from the above code is a the PDF setanim.pdf_ where each page contains a frame of the animation.

Step 2. Putting together the frames
-----------------------------------

The next step is to put the frames together and create an animation. There are different ways you can do this, but I will only demonstrate the `animate.sty`_ package. Some alternatives are:

- Convert the pages in the PDF to bitmaps and put them together using video creation tools like ffmpeg_, Blender_ and many others.
- Use SWFTools_' pdf2swf_ tool to create a Flash animation.

.. _ffmpeg: http://ffmpeg.mplayerhq.hu/
.. _Blender: http://www.blender.org/
.. _SWFTools: http://www.swftools.org/about.html
.. _pdf2swf: http://www.swftools.org/pdf2swf.html


The animate package provides the command ``\animatedgraphics`` for creating animations based on a sequence of files or a multi-page PDF:

.. sourcecode:: latex

    \animategraphics[<options>]{<frame rate>}{<file basename>}{<first>}{<last>}

See the documentation_ for an explanation of the available options and parameters. Here is the code for including the the frames:

.. sourcecode:: latex

    \documentclass{beamer}

    \usepackage{animate}
    \begin{document}

    \begin{frame}
        \frametitle{Set intersection}
        \begin{center}
            % Set frame rate to 12 frames/sec. Load frames from setanim.pdf.
            % We will use all frames, so there is no need to specify a start
            % and end frame.
            \animategraphics[autoplay,palindrome]{12}{animations/setanim}{}{}
        \end{center}
    \end{frame}

    \end{document}

.. _documentation: http://www.ctan.org/get/macros/latex/contrib/animate/doc/animate.pdf

Download the PDF to see the animation in action.

:Source: The latex-beamer-users_ mailing list

.. _latex-beamer-users: http://www.nabble.com/Re%3A--animate.sty--simple-example-p15390560.html

.. note:: Acrobat Reader v >= 6.0 is required to view this example.

.. _animate.sty: http://www.ctan.org/tex-archive/help/Catalogue/entries/animate.html
.. _animate: http://www.ctan.org/tex-archive/help/Catalogue/entries/animate.html

A standalone version
--------------------

The author of the animate_ package, Alexander Grahn, has contributed a version of the above example that does not require splitting the frames and the viewer in separate files:

- Source: `animated-set-standalone.tex`_
- Output: `animated-set-standalone.pdf`_

.. _animated-set-standalone.tex: animations/animated-set-standalone.tex
.. _animated-set-standalone.pdf: animations/animated-set-standalone.pdf

.. sourcecode:: latex

    \documentclass{beamer}
    \usepackage{tikz}
    \usepackage{animate}

    \usepackage{fp}

    % parameterized tikz graphics
    \newcommand{\intersect}[1]{%
    \def\setA{(0,0) circle (1)}%
    \def\setB{(#1,0) circle (1)}%
    % define the bounding box
    \def\boundb{(-2,2) rectangle (4,-2)}%
    %
    \begin{tikzpicture}
        \draw \boundb;
        % intersection
        \begin{scope}
        \clip \setA;
        \fill[black!20] \setB;
        \end{scope}
        \begin{scope}[even odd rule]% first circle without the second
        \clip \setB \boundb;
        \fill[red!20] \setA;
        \end{scope}
        \begin{scope}[even odd rule]% first circle without the second
        \clip \setA \boundb;
        \fill[blue!20] \setB;
        \end{scope}
        \draw \setA;
        \draw \setB;
        \node at (-1,0) [left] {$A$};
        \node at (#1+1,0) [right] {$B$};
        \node at (4,2) [below left] {$A\cap B$};
    \end{tikzpicture}
    }

    \begin{document}

    \begin{frame}
    \frametitle{Set intersection}
    \begin{center}

        \begin{animateinline}[autoplay,palindrome]{12}
        %first frame, xb=0.0
        \gdef\xb{0}% xb initial value
        \intersect{\xb}%
        %remaining frames, xb=0.1...2.1
        \whiledo{\lengthtest{\xb pt < 2.1pt}}{%
            \newframe
            \FPeval{xb}{\xb+0.1}% new xb
            \xdef\xb{\xb}% make \xb global
            \intersect{\xb}%
        }%
        \end{animateinline}

    \end{center}
    \end{frame}
    \end{document}


\end{comment}

\begin{document}

\begin{frame}
    \frametitle{Set intersection}
    \begin{center}
        % Set frame rate to 12 frames/sec. Load frames from setanim.pdf.
        % We will use all frames, so there is no need to specify a start
        % and end frame.
        \animategraphics[autoplay,palindrome]{12}{animations/setanim}{}{}
    \end{center}
\end{frame}
\end{document}