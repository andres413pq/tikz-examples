\documentclass{article}

\usepackage[latin1]{inputenc}
\usepackage{tikz}
\usetikzlibrary{patterns}

%%%<
\usepackage{verbatim}
\usepackage[active,tightpage]{preview}
\PreviewEnvironment{tikzpicture}
\setlength\PreviewBorder{5pt}%
%%%>

\begin{comment}
:Title: Three link manipulator
:Tags: Macros, Physics & chemistry

One of the nicest things about programming illustrations, is that it's easy
to change parameters. In this example I've parameterized a three link manipulator
using a macro. I can then draw the manipulator in different positions with just one
line of code.

\end{comment}


\begin{document}
\pagestyle{empty}

% Note. This illustration was originally made with PSTricks. Conversion to
% PGF/TikZ was straightforward. However, I could probably have made it more
% elegant.

% Define a variable as a length
\newcommand{\nvar}[2]{%
    \newlength{#1}
    \setlength{#1}{#2}
}

% Define a few constants for drawing
\nvar{\dg}{0.3cm}
\def\dw{0.25}\def\dh{0.5}
% Define commands for links, joints and such
\def\link{\draw [double distance=1.5mm, very thick] (0,0)--}
\def\joint{%
    \filldraw [fill=white] (0,0) circle (5pt);
    \fill[black] circle (2pt);
}
\def\grip{%
    \draw[ultra thick](0cm,\dg)--(0cm,-\dg);
    \fill (0cm, 0.5\dg)+(0cm,1.5pt) -- +(0.6\dg,0cm) -- +(0pt,-1.5pt);
    \fill (0cm, -0.5\dg)+(0cm,1.5pt) -- +(0.6\dg,0cm) -- +(0pt,-1.5pt);
}

\def\robotbase{%
    \draw[rounded corners=8pt] (-\dw,-\dh)-- (-\dw, 0) --
        (0,\dh)--(\dw,0)--(\dw,-\dh);
    \draw (-0.5,-\dh)-- (0.5,-\dh);
    \fill[pattern=north east lines] (-0.5,-1) rectangle (0.5,-\dh);
}

% This macro draws a three link manipulator.
% Input parameters:
%   #1 theta_1
%   #2 L_1
%   #3 theta_2
%   #4 L_2
%   #5 theta_3
%   #6 L_3
%
% Example:
%   \threelink{60}{2}{-70}{2}{30}{1}
\newcommand{\threelink}[6]{%
    \robotbase
    \link(#1:#2);
    \joint
    \begin{scope}[shift=(#1:#2), rotate=#1]
        \link(#3:#4);
        \joint
        \begin{scope}[shift=(#3:#4), rotate=#3]
            \link(#5:#6);
            \joint
            \begin{scope}[shift=(#5:#6), rotate=#5]
                \grip
            \end{scope}
        \end{scope}
    \end{scope}
}

\begin{tikzpicture}
    \threelink{60}{2}{-90}{2}{-60}{1}
    \begin{scope}[xshift=4cm]
        \threelink{60}{2}{-110}{2}{90}{1}
    \end{scope}
    \begin{scope}[shift={(2cm, -3.2cm)}]
        % Illustration of two different solutions to the inverse kinematic
        % problem.
        \begin{scope}[dashed]
            \threelink{-10}{2}{70}{2}{-40}{1}
        \end{scope}
        \threelink{60}{2}{-70}{2}{30}{1}
    \end{scope}
\end{tikzpicture}

\end{document}