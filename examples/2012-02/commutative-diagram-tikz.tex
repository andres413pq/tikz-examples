% A simple commutative diagram
% Stefan Kottwitz
\documentclass{article}
\usepackage{tikz}
%%%<
\usepackage{verbatim}
\usepackage[active,tightpage]{preview}
\PreviewEnvironment{tikzpicture}
\setlength\PreviewBorder{5pt}%
%%%>
\begin{comment}
:Title: Commutative diagram
:Tags: Matrices;Arrows;Diagrams;Mathematics
:Author: Stefan Kottwitz
:Slug: commutative-diagram-tikz

A simple example of a commutative diagram using TikZ, short and readable.
It has been posted as answer to the question
http://tex.stackexchange.com/q/45741/213 of Elias.

* A matrix is used for positioning the main nodes
* Arrows are drawn as edges, between the main nodes,
  using further nodes for labeling
* As columns and row distance is expressed by em units, which scale
  with the font size, the diagram can be scaled by using font size commands
  such as \Large, \huge etc.

Note the (m-2-1.east|-m-2-2) syntax for getting a horizontal arrow
between cells of a different height.
\end{comment}
\usetikzlibrary{matrix}
\begin{document}
\begin{tikzpicture}
  \matrix (m) [matrix of math nodes,row sep=3em,column sep=4em,minimum width=2em] {
     F_t(x) & F(x) \\
     A_t & A \\};
  \path[-stealth]
    (m-1-1) edge node [left] {$\mathcal{B}_X$} (m-2-1)
            edge [double] node [below] {$\mathcal{B}_t$} (m-1-2)
    (m-2-1.east|-m-2-2) edge node [below] {$\mathcal{B}_T$} node [above] {$\exists$} (m-2-2)
    (m-1-2) edge node [right] {$\mathcal{B}_T$} (m-2-2)
            edge [dashed,-] (m-2-1);
\end{tikzpicture}
\end{document}

