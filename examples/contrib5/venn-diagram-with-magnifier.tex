\documentclass{minimal}

\usepackage{tikz}
\usetikzlibrary{spy}

\usepackage{verbatim}
\usepackage[active,tightpage]{preview}
\PreviewEnvironment{tikzpicture}

\begin{document}
\pagestyle{empty}

\begin{comment}
:Title: Venn diagram with magnifier
:Tags:  spy, Magnifier, Transparency

This example shows how to add a glass to magnify a special part of a pictures.
It makes use of the new spy library, so you'll need a recent TikZ version [1]
to compile it.

It was created by Dennis Heidsiek [2], based on the example [3],
after seeing [4].

[1] http://www.texample.net/tikz/builds/
[2] http://www.google.com/profiles/Dennis.Heidsiek
[3] http://www.texample.net/tikz/examples/venn-diagram/
[4] http://wiki.the-big-bang-theory.com/wiki/Psychic_Vortex
\end{comment}

% First, we define three circles:
\def\firstcircle{(-2,0) circle (2.4)}
\def\secondcircle{(2,0) circle (2.4)}
\def\h{1.3266499161421599396459730946683} % h=(2.4^2-2^2)^(1/2) by pythagoras
\def\thirdcircle{(0,-\h) circle (0.2cm)}

\begin{tikzpicture}
 % Let's draw the scene (to magnify):
  \begin{scope}[spy using outlines=
      {magnification=16, size=8cm, connect spies, rounded corners}]
      
    % the boarder:
    \draw[thick, rounded corners] (-5,-4) rectangle (5,4);
    \draw (0,3.4) node[scale=2] {the universe of all women};
    
    % The transparency:
    \begin{scope}[fill opacity=0.5]
      \fill[red] \firstcircle;
      \fill[green] \secondcircle;
      \fill[blue] \thirdcircle;
    \end{scope}
    
    % letterings and missing pieces:
    \draw[align=center] \firstcircle
        node {woman you\\want to\\sleep with};
    \draw[align=center] \secondcircle
        node {women who\\believe exactly\\what you believe};
    \draw \thirdcircle;
    \draw (0,-2.3)
        node[align=center] {wo-\\men\\who would be willing to sleep with you};
    \fill(0,-\h+0.12) circle (0.005)
        node[scale=0.11, align=center] {ideal\\mate};
    
    % now we can draw the magnifying glass:
    \spy [red] on (0,-\h) in node [left] at (13.25,0);
  \end{scope}
\end{tikzpicture}

\end{document}