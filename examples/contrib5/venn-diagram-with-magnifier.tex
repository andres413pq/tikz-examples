% Venn diagram with magnifier
% Author: Dennis Heidsiek
\documentclass{minimal}

\usepackage{tikz}
\usetikzlibrary{spy}

\usepackage{verbatim}
\usepackage[active,tightpage]{preview}
\PreviewEnvironment{tikzpicture}

\begin{document}
\pagestyle{empty}

\begin{comment}
:Title: Venn diagram with magnifier

This example shows how to add a glass to magnify a special part of a pictures.
It makes use of the new spy library, so you'll need a recent TikZ version [1]
to compile it.

It was created by Dennis Heidsiek [2], based on the example [3],
inspired by a sketch in [4].

- [1] http://www.texample.net/tikz/builds/
- [2] http://www.google.com/profiles/Dennis.Heidsiek
- [3] http://www.texample.net/tikz/examples/venn-diagram/
- [4] http://wiki.the-big-bang-theory.com/wiki/Psychic_Vortex

\end{comment}

% First, we define three circles:
\def\firstcircle{(-2,0) circle (2.4)}
\def\secondcircle{(2,0) circle (2.4)}
\pgfmathparse{-(2.4^2-2^2)^0.5} % by pythagoras
\let\h\pgfmathresult % shortcut for further use
\def\thirdcircle{(0,\h) circle (0.2cm)}

\begin{tikzpicture}
 % Let's draw the scene (to magnify):
  \begin{scope}[spy using outlines=
      {magnification=16, size=8cm, connect spies, rounded corners}]
    
    % the boarder:
    \draw[thick, rounded corners] (-5,-4) rectangle (5,4);
    \draw (0,3.3) node[scale=2] {the universe of all jobs};
    
    % The transparency:
    \begin{scope}[fill opacity=0.5]
      \fill[red] \firstcircle;
      \fill[green] \secondcircle;
      \fill[blue] \thirdcircle;
    \end{scope}
    
    % letterings and missing pieces:
    \draw[align=center] \firstcircle node {what you\\would like\\to do};
    \draw[align=center] \secondcircle node {what you\\are able\\to do};
    \draw \thirdcircle
      (0,-2.3) node[align=center] {em-\\ployment\\you get paid enough for};
    \fill (0,\h+0.12) circle (0.005)
        node[scale=0.11, align=center] {perfect\\job};
    
    % now we can draw the magnifying glass:
    \spy [red] on (0,\h) in node [left] at (13.25,0);
  \end{scope}
\end{tikzpicture}

\end{document}