\documentclass{article}
\usepackage{tikz}
\usepackage{ifthen}
\usepackage{amsmath}
\usetikzlibrary{arrows,calc,intersections}

% 1) Represent n points on a cercle
% 2) Draw the complete graph on these points 
% 3) Draw all the intersection points of any two of these segments
\begin{document}
	\def\r{4}
	\def\n{8} \def\myangles{{25,50,85,125,160,220,250,280,340}}
	%This vector contains the angles determining the position of the points.
	%-----------------------------------------------------------
	% Variables and counters used to generate the 4-combinations
	\newcounter{np} \pgfmathsetcounter{np}{\n+1}
	\newcounter{na} \newcounter{nb} \newcounter{nc}
	\newcounter{ia} 
	\pgfmathsetcounter{na}{\n-1}	% saves some computations later
	\pgfmathsetcounter{nb}{\n-2}	% ""
	\pgfmathsetcounter{nc}{\n-3}	%	""
	\newcounter{q} \setcounter{q}{0}	% if flag q=1 then exit the whiledo loop
	\newcounter{e} \setcounter{e}{0}	% e counts the combinations!
	\newcounter{a} \setcounter{a}{0}	% element of the 4-combination
	\newcounter{b} \setcounter{b}{1}	% ""
	\newcounter{c} \setcounter{c}{2}	% ""
	\newcounter{d} \setcounter{d}{2}	% ""
	%Watch out! The initial value {0,1,2,2} is not a 4-combination
	%-----------------------------------------------------------
	Consider $n$ randomly placed points on a circle.
	\begin{enumerate}
		\item The complete graph on the $n$ points has $\begin{pmatrix}n\\2\end{pmatrix}$ edges.
		\item Each pair of edges yields an intersection point and there are (at most) $\begin{pmatrix}n\\4\end{pmatrix}$ such points.
	\end{enumerate}
	
	\begin{center}
	\begin{tikzpicture}
		% Draw the complete graph
		\fill[fill=blue!10!green!10!,draw=blue,dotted,thick] (0,0) circle (\r);
		\pgfmathparse{\n-1} \let\nn\pgfmathresult 
		\foreach \i in {0,...,\nn}{
			\pgfmathparse{\i+1} \let\ii\pgfmathresult
			\pgfmathparse{\myangles[\i]} \let\t\pgfmathresult
			\foreach \j in {\ii,...,\n}
				\pgfmathparse{\myangles[\j]} \let\u\pgfmathresult
				\draw[blue,very thick] ({\r*cos(\t)},{\r*sin(\t)})--({\r*cos(\u)},{\r*sin(\u)});
			}
		\foreach \i in {0,...,\n}{
			\pgfmathparse{\myangles[\i]}	\let\t\pgfmathresult
			\pgfmathsetcounter{ia}{\i+1}
			\fill[draw=blue,fill=blue!20!,thick]
					({\r*cos(\t)},{\r*sin(\t)})circle (2.5mm) node{$\mathbf{\theia}$};
			}
			% Points and segments are now drawn
			%
		\whiledo{\theq=0}{ % this loop generates the 4-combinations
			\stepcounter{e}
			\ifthenelse{\thee=1000}{\setcounter{q}{1}}{}% just to be sure to get out of the loop some day...
			\ifthenelse{\thed=\n}
				{\ifthenelse{\thec=\thena}
					{\ifthenelse{\theb=\thenb}
						{\ifthenelse{\thea=\thenc}
							{\setcounter{q}{1}}
							{	\stepcounter{a}
								\pgfmathsetcounter{b}{\thea+1}
								\pgfmathsetcounter{c}{\thea+2}
								\pgfmathsetcounter{d}{\thea+3}					
							}
						}
						{	\stepcounter{b}
							\pgfmathsetcounter{c}{\theb+1}
							\pgfmathsetcounter{d}{\theb+2}						
						}
					}
					{	\stepcounter{c}
						\pgfmathsetcounter{d}{\thec+1}
					}
				}
				{\stepcounter{d}}
			\ifthenelse{\theq=0}{
				% Construction of the intersection points of the segments
				\pgfmathparse{\r*cos(\myangles[\thea])} \let\xa\pgfmathresult
				\pgfmathparse{\r*sin(\myangles[\thea])} \let\ya\pgfmathresult
				\pgfmathparse{\r*cos(\myangles[\theb])} \let\xb\pgfmathresult
				\pgfmathparse{\r*sin(\myangles[\theb])} \let\yb\pgfmathresult
				\pgfmathparse{\r*cos(\myangles[\thec])} \let\xc\pgfmathresult
				\pgfmathparse{\r*sin(\myangles[\thec])} \let\yc\pgfmathresult
				\pgfmathparse{\r*cos(\myangles[\thed])} \let\xd\pgfmathresult
				\pgfmathparse{\r*sin(\myangles[\thed])} \let\yd\pgfmathresult
				%				
				\coordinate  (A) at (\xa,\ya);
				\coordinate  (B) at (\xb,\yb);
				\coordinate  (C) at (\xc,\yc);
				\coordinate  (D) at (\xd,\yd);
				% Name the coordinates, but do not draw anything!				
				\path[name path=sega] (A) -- (C);
				\path[name path=segb] (B) -- (D);
				\path [name intersections={of=sega and segb}];
				\coordinate (X) at (intersection-1);
				\fill[fill=green!50!,draw=blue] (X) circle (0.8mm);				
				%-----------------------------------------------------
				}{}
		}% End of the whiledo loop
	\end{tikzpicture}
	\end{center}
	\addtocounter{e}{-1}
	Number of generated intersection points : \thee
\end{document}