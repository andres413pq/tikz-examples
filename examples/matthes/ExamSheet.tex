% Exam sheet
% Author: Alain Matthes
% Requires pdflatex and two compilations to place the overlays correctly
\documentclass{article}
\usepackage{tikz}

%%%<
\usepackage{verbatim}
%%%>

\begin{comment}
:Title: Exam sheet
:Tags: Overlays, Transformations

Overlays and the special ``current page`` node offers precise control of 
placement on a page. One application is to create exam sheets where candidate details 
can be hidden using a `dog ears`_ technique. 

:Author: `Alain Matthes`_

.. _Alain Matthes: http://www.altermundus.fr/index.html
.. _dog ears: http://www.answers.com/topic/dog-ear

\end{comment}

\begin{document}
 \thispagestyle{empty}
\begin{tikzpicture}[remember picture, overlay]
    \node [shift={(-8 cm,-8cm)}]  at (current page.north east)
        {%
        \begin{tikzpicture}[remember picture, overlay]
            \draw(0,0) rectangle (8,8);
            \draw [fill=gray] (0,8) -- (8,8) -- (8,0) -- cycle ;
            \path[coordinate]
                \foreach \k in {1,...,4}{%
                    (0 pt,8cm -\k *1.2cm) coordinate (d\k)} ;
            \path[clip] (0,0) rectangle (8,8);
            \foreach \k/\t in {1/Last Name,2/First Name,3/Birthday,4/Date}{%
                \node[inner sep=0pt,rotate=-45,%
                    right=0.5cm,minimum height=12pt](f\k) at (d\k) {\t};
                \draw  (f\k.south east)-- (8cm -\k * 1.2cm,-6pt );
            }
        \end{tikzpicture}
        };
        % Draw a page border
        \draw (current page.north west) rectangle (current page.south east);
\end{tikzpicture}


\end{document}

