% Author: Alain Matthes (http://altermundus.com/)
\documentclass[a4paper,11pt]{article}
\usepackage[utf8]{inputenc}
\usepackage[upright]{fourier}
\usepackage{amsmath,tkz-linknodes}
\usepackage{tikz}
%%%<
\usepackage{verbatim}
\usepackage[active,tightpage]{preview}
\PreviewEnvironment{tikzpicture}
\setlength\PreviewBorder{5pt}%
%%%>

\begin{comment}
:Title: Tkz-linknodes examples

Demonstration of the `tkz-linknodes package`_, a package that makes it easy to link
elements in an amsmath align or aligned environment.

.. _tkz-linknodes package: http://www.ctan.org/tex-archive/help/Catalogue/entries/tkz-linknodes.html

\end{comment}
\begin{document}

\parindent=0pt
\begin{preview}
\begin{center}
  \fbox{%
  \begin{minipage}{10cm}
        \begin{NodesList}[margin=2 cm]
       \begin{align}
     3\left(x^2-\frac{2}{3}\right) &= 4                             \AddNode\\
       3x^2-2  &= 4                                                 \AddNode\\
       3x^2    &= 6                                                 \AddNode\\
       \intertext{\hfil isolate the term with the variable \hfil}
        x^2    &= 2                                                 \AddNode\\
 \sqrt{x^2}    &= \sqrt{2}                                          \AddNode\\
      |x|      &= \sqrt{2}                                          \AddNode\\
       x       &= \pm\sqrt{2}                                       \AddNode
         \end{align}
 \LinkNodes{expand}%
 \LinkNodes{$+2$}%
 \LinkNodes{$\div 3$}
 \LinkNodes{$\sqrt{\ldots}$}
 \LinkNodes{$\sqrt{x}=|x|$}
 \LinkNodes{so that}
   \end{NodesList}
  \end{minipage}}
\end{center}

This example is from  MathMode.pdf of Herbert Vo\ss

\begin{NodesList}[margin=1cm]
  \begin{displaymath}\displaywidth=.2\linewidth
    \begin{aligned}
    y &= 2x^2 -3x +5                          \AddNode\\
  & \hphantom{= \ 2\left(x^2-\frac{3}{2}\,x\right. }%
      \textcolor{blue}{%
        \overbrace{\hphantom{+\left(\frac{3}{4}\right)^2- %
          \left(\frac{3}{4}\right)^2}}^{=0}}   \\
  &= 2\left(\textcolor{red}{%
       \underbrace{%
           x^2-\frac{3}{2}\,x + \left(\frac{3}{4}\right)^2}%
   }%
   \underbrace{%
        - \left(\frac{3}{4}\right)^2 + \frac{5}{2}}%
   \right)                                      \AddNode\\
   &= 2\left(\qquad\textcolor{red}{\left(x-\frac{3}{4}\right)^2}
   \qquad + \ \frac{31}{16}\qquad\right)  \AddNode\\
y
   &= 2\left(x\textcolor{cyan}{-\frac{3}{4}}\right)^2\textcolor{blue}{+\frac{31}{8}}\AddNode
\end{aligned}
         \end{displaymath}
{%
\tikzset{LabelStyle/.append style = {left,text=red}}
    \LinkNodes{%
    \begin{minipage}{5cm}
      $2x^2 -3x$ is the beginning of an algebraic identity %
       (binomial formula)
      \end{minipage}}
       \LinkNodes{$(a-b)^2=a^2-2ab+b^2$}
       \LinkNodes{after simplication, the result is}%
}
\end{NodesList}
\end{preview}


\end{document} 