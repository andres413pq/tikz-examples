% The tkz-berge graph package
% Author: Alain Matthes (http://altermundus.fr/)

\documentclass[]{article}
\usepackage{tikz}
\usepackage{tikz,fullpage}
\usetikzlibrary{arrows,%
                petri,%
                topaths}%
\usepackage{tkz-berge}
\usepackage[position=top]{subfig}

%%%<
\usepackage{verbatim}
\usepackage[active,floats,tightpage]{preview}
\PreviewEnvironment{tikzpicture}
\setlength\PreviewBorder{5pt}%
%%%>

\begin{comment}
:Title: The tkz-berge graph package
:Slug: tkz-berge
:Tags: Graphs

The ``tkz-berge`` package is a set of convenient macros for drawing typical textbook graphs.
It is built on top of TikZ' node and edge constructs, but adds several shortcuts for placing, 
arranging  and connecting vertices. Visit the package's home page to see  more_ examples_.

Documentation_ only available in French. However, the code and the numerous examples speak for themselves.

:Download: `tkz-berge.sty`_
:Author: `Alain Matthes`_
:Source: `Altermundus`_

.. _Alain Matthes: http://altermundus.fr/index.html
.. _Altermundus: http://www.altermundus.fr/pages/pdflatex/graph.html
.. _more: http://www.altermundus.fr/pages/pdflatex/graph.html
.. _examples: http://www.altermundus.fr/pages/math/graphtheory.html
.. _Documentation: http://www.altermundus.fr/pages/downloads/doc-TKZberge.pdf
.. _tkz-berge.sty: http://www.altermundus.fr/pages/downloads/tkz-berge.sty

\end{comment}

\begin{document}
\begin{figure}
\centering
\subfloat[]{
\begin{tikzpicture}[scale=0.75,transform shape]
  \Vertex[x=0,y=0](K)
  \Vertex[x=0,y=2](F)
  \Vertex[x=-1,y=4](D)
  \Vertex[x=3,y=7](H)
  \Vertex[x=8,y=5](B)
  \Vertex[x=9,y=2](N)
  \Vertex[x=5,y=0](M)
  \Vertex[x=3,y=1](S)
  \tikzstyle{LabelStyle}=[fill=white,sloped]
  \tikzstyle{EdgeStyle}=[bend left]
  \Edge[label=$120$](K)(F)
  \Edge[label=$650$](H)(S)
  \Edge[label=$780$](H)(M)
  \Edge[label=$490$](D)(B)
  \Edge[label=$600$](D)(M)
  \Edge[label=$580$](B)(M)
  \Edge[label=$600$](H)(N)
  \Edge[label=$490$](F)(H)
  \tikzstyle{EdgeStyle}=[bend right]
  \Edge[label=$630$](S)(B)
  \Edge[label=$210$](S)(N)
  \Edge[label=$230$](S)(M)
\end{tikzpicture}
}
\subfloat[]{
\begin{tikzpicture}[scale=0.75,transform shape]
  \tikzstyle{every node}=[node distance = 4cm,%
                          bend angle    = 45,%
                          fill          = gray!30]
  \Vertex(P)
  \NOEA{P}(B)
  \SOEA{P}(M)
  \NOEA{B}(D)
  \SOEA{B}(C)
  \SOEA{C}(L)
  \tikzstyle{EdgeStyle}=[pre and post]
  \Edge[label=$4$](P)(M)
  \Edge[label=$9$](C)(M)
  \Edge[label=$4$](C)(L)
  \Edge[label=$5$](C)(D)
  \Edge[label=$10$](B)(M)
  \tikzstyle{EdgeStyle}=[pre and post,bend right]
  \Edge[label=$11$](L)(D)
  \tikzstyle{EdgeStyle}=[post]
  \Edge[label=$3$](C)(B)
  \Edge[label=$10$](D)(B)
  \Edge[label=$10$](L)(M)
  \Edge[label=$10$](B)(P)
\end{tikzpicture}
}
\end{figure}
\end{document}
