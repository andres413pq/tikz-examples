\documentclass{article}
\usepackage[symbols]{circuitikz}
\usepackage{tikz}

%%%<
\usepackage{verbatim}
\usepackage[active,tightpage]{preview}
\PreviewEnvironment{circuitikz}
\setlength\PreviewBorder{0pt}%
%%%>

\begin{comment}
:Title: Circuitikz
:Grid: 2x2


CircuiTikz_ is a set of LaTeX macros designed to make it easy to draw electrical networks in scientific publications. It provides a convenient syntax based on to-paths to place the various components. 
The examples below are from the `CircuiTikz examples page`_. The author of CircuiTikz_ is `Massimo Readelli`_.

To run the examples you need to `download and install`_ the CircuiTikz_ files first. 

*Note*. The circuits library available in the CVS version of PGF is inspired by CircuiTikz_. 

.. _CircuiTikz: http://home.dei.polimi.it/mredaelli/circuitikz/index.html
.. _CircuiTikz examples page: http://home.dei.polimi.it/mredaelli/circuitikz/examples.html
.. _download and install: http://home.dei.polimi.it/mredaelli/circuitikz/download.html
.. _Massimo Readelli: http://home.dei.polimi.it/mredaelli/about.html
\end{comment}

\begin{document}

\begin{circuitikz} \draw
 (0,0) to[C, l=$10\micro\farad$] (0,2) -- (0,3)
  to[R, l=$2.2\kilo\ohm$] (4,3) -- (4,2)
  to[L, l=$12\milli\henry$, i=$i_1$] (4,0) -- (0,0)
 (4,2) to[D*, *-*] (2,0) to [D*, -*] (0,2)
  to[R, l=$1\kilo\ohm$] (2,2)   to[cV, v=$0.3\kilo\ohm i_1$] (4,2)
 (2,0) to[I, i=$1\milli\ampere$:15, -*] (2,2)
; 
\end{circuitikz}

\begin{circuitikz} \draw
 (0,0) node[ground] {}
  to[V, v=$e(t)$, *-*] (0,2) to[C, l=$4\nano\farad$] (2,2)
  to [R, l=$\frac{1}{4}\kilo\ohm$, *-*] (2,0)
 (2,2) to[R, l=$1\kilo\ohm$] (4,2)
  to[C, l=$2\nano\farad$:-90, *-*] (4,0)
 (5,0) to[I, i=$a(t)$:-90, -*] (5,2) -- (4,2)
 (0,0) -- (5,0)
 (0,2) -- (0,3) to[L, l=$2\milli\henry$] (5,3) -- (5,2)

 {[anchor=south east] (0,2) node {1} (2,2) node {2} (4,2) node {3}}
;\end{circuitikz}

\begin{circuitikz} \draw
 (0,0) node[anchor=east]{B}
  to[short, o-*] (1,0)
  to[R, l=$20\ohm$, *-*] (1,2)
  to [R, v=$v_x$, l=$10\ohm$] (3,2)
  to[short] (4,2) to[cI, i=$\frac{\siemens}{5}v_x$, *-*] (4,0)
   to[short] (3,0) to[R, l=$5\ohm$, *-*] (3,2)
 (3,0) -- (1,0)
 (1,2) to[short, *-o] (0,2)
  node[anchor=east]{A}
;\end{circuitikz}

\begin{circuitikz} \draw
 (0,2) to[I, i=$1\milli\ampere$] (2,2)
  to [R, l=$2\kilo\ohm$:-90, *-*] (0,0)
  to [R, l=$2\kilo\ohm$] (2,0)
  to[V, v=$2\volt$:-90] (2,2)
  to[cspst, l=$t_0$] (4,2) -- (4,1.5)
  to [generic, l=1, i=$i_1$, v=$v_1$] (4,-.5) -- (4,-1)
 (0,2) -- (0,-1) to[V, v=$4\volt$] (2,-1)
  to [R, l=$1\kilo\ohm$] (4,-1);

 \begin{scope}[xshift=7.5cm, yshift=.5cm]
  \draw [->] (-2,0)--(2.5,0) node[anchor=west] {$v_1 [\volt]$};
  \draw [->] (0,-2)--(0,2) node[anchor=west] {$i_1 [\milli\ampere]$} ;
  \draw (-1,0) node[anchor=north] {-2} (1,0) node[anchor=south] {2}
   (0,1) node[anchor=west] {4} (0,-1) node[anchor=east] {-4} (2,0) 
   node[anchor=north west] {4} (-1.5,0) node[anchor=south east] {-3};
  \draw [thick]
   (-2,-1) -- (-1,1) -- (1,-1) -- (2,0) -- (2.5,.5);
  \draw [dotted]
   (-1,1) -- (-1,0) (1,-1) -- (1,0) (-1,1) -- (0,1) (1,-1) -- (0,-1);
 \end{scope}
\end{circuitikz}

\end{document} 