% Intersecting rings
% Author: Dominique W�rtz
% Source: http://www.nabble.com/Intersection-area-to15711124.html

\documentclass{minimal}

\usepackage{tikz}

%%%<
\usepackage{verbatim}
\usepackage[active,tightpage]{preview}
\PreviewEnvironment{tikzpicture}
\setlength\PreviewBorder{5pt}%
%%%>

\begin{comment}
:Title: Intersecting rings
:Tags: Clipping

This example shows how to fill the intersection area of two rings. 
The operation requires you to apply the ``even odd rule`` to the clipping path. Unfortunately
you can not set some options directly when specifying the clipping path. You can easily solve
this problem by putting the path inside a scope. (See Chapter 14.7 in the manual for an
explanation)

:Author: Dominique W�rtz
:Source: `pgf-users mailing list`_

.. _pgf-users mailing list: http://www.nabble.com/Intersection-area-to15711124.html

\end{comment}

\begin{document}
 
% Define the rings. Store them in macros to make things
% more flexible. 
\def\ringa{(-1,0) circle (2) (-1,0) circle (3)}
\def\ringb{(1,0) circle (2) (1,0) circle (3)}

\begin{tikzpicture}
    % First we fill the intersecting area
    % The \clip command does not allow options, therefore 
    % we have to use a scope to set the even odd rule. 
    \begin{scope}[even odd rule]
        % Define a clipping path. All paths outside ringa will
        % be cut because the even odd rule is set. 
        \clip \ringa;
        % Fill ringb. Since the even odd rule is set, only the
        % ring will be filled, not the hole in the middle.  
        \fill[fill=orange] \ringb;
    \end{scope}
    % Then we draw the rings
    \draw \ringa;
    \draw \ringb;
\end{tikzpicture}

\end{document}