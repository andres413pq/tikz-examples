% Electrical circuits using decorations
% Author: B. Umesh Rai
\documentclass{article}%

\usepackage{electComp}
\usetikzlibrary{decorations,decorations.pathmorphing,decorations.pathreplacing}

%%%<
\usepackage{verbatim}
\usepackage[floats,active,tightpage]{preview}
\setlength\PreviewBorder{5pt}%
\PreviewEnvironment{tikzpicture}
%%%>

\begin{comment}
:Title: Electric circuit decorations
:Slug: circuit-decorations
:Tags: Decorations, Diagrams

This example demonstrates an interesting use of decorations. A set of custom decorations
have been defined to draw various electrical circuit elements. The code for the decorations 
can be found in the `electComp.sty`_ file.

To compile the example you will need the following style file:

- `electComp.sty`_

.. _electComp.sty: tzbondgraph/electComp.sty

:Author: B\. Umesh Rai

\end{comment}

\begin{document}
\title{Electrical Circuits with pgf}

\date{}
\maketitle
%-------------------------------------------

\begin{tikzpicture}[line width=1pt]
    \draw (0,0) -- ++(0,1cm);
    \draw[decorate, decoration=cell] (0,1cm) -- ++(0,1.5cm);
    \draw (0,2.5cm) |- ++(1cm,1cm);
    \draw[decorate, decoration=diode] (1cm,3.5cm) -- ++(1.5cm,0);
    \draw (2.5cm,3.5cm) -- ++(2,0);
    \draw (3.5cm,3.5cm) -- ++(0,-1);
    \draw[decorate, decoration=switch] (3.5cm,2.5cm) -- ++(0,-1.5cm);
    \draw[decorate, 
        decoration={inductor,amplitude=0.35cm, segment length=0.75cm}]
        (4.5cm,3.5cm) -- ++(1.5cm,0);
    \draw (6cm,3.5cm) -- ++(1,0);
    \draw (7cm,3.5cm) -- ++(3,0);
    \draw (8cm,3.5cm) -- ++(0,-1);
    \draw[decorate, decoration=capacitor] (8cm,2.5cm) -- ++(0,-1.5cm);
    \draw (0cm,0cm) -| ++(8cm,1cm);
    \draw (3.5cm,0cm) -- ++(0cm,1cm);
    \draw (10cm,3.5cm) -- ++(0,-1);
    \draw[decorate, decoration=resistor] (10cm,2.5cm) -- ++(0,-1.5cm);
    \draw (8cm,0cm) -| ++(2cm,1cm);
    \draw[decorate, decoration=ground] (5,0cm) -- ++(0,-1.5cm);
\end{tikzpicture}

\end{document}


