% Daniell's Pile
% Author: Agustin E. Bolzan
\documentclass{article}
\usepackage{tikz}

%%%<
\usepackage{verbatim}
\usepackage[active,tightpage]{preview}
\PreviewEnvironment{tikzpicture}
\setlength\PreviewBorder{5pt}%
%%%>

\begin{comment}
:Title: Daniell's pile
:Tags: Physics & chemistry, Transparency

This is an illustration of the famous Daniell's pile. Note how transparency has been 
used to create an illusion of depth. To get the correct drawing order, the front and
the back of the vessels had to be drawn separately.

| Author: Agustin E. Bolzan 

\end{comment}

\begin{document}

\definecolor{copper}{cmyk}{0,0.9,0.9,0.2}
\colorlet{lightgray}{black!25}
\colorlet{darkgray}{black!75}

\centering
\Large {Daniell's Pile}
\vspace{2cm}


\begin{tikzpicture}
    % Draw back of vessel 1
    \draw (0,0) to [controls=+(90:0.5) and +(90:0.5)] (2,0);
    \draw[fill=blue!60, fill opacity=0.5] (0,-0.5) to
        [controls=+(90:0.5) and +(90:0.5)] (2,-0.5);

    % Draw back of vessel 2
    \draw (3.5,0) to [controls=+(90:0.5) and +(90:0.5)] (5.5,0);
    \draw[fill=lightgray, fill opacity=0.5] (3.5,-0.5) to [controls=+(90:0.5)
        and +(90:0.5)] (5.5,-0.5);

    % Draw copper electrode
    \draw[fill=copper] (0.5,2) rectangle (1.5,-1);
    \draw (0,2.3) node {Cu};
    \draw (1,-1.75) node {\small{CuSO$_{4}$}};

    % Draw salt bridge

    \draw[join=round, line width = 10pt] (1.75,-1.75) -- (1.75,0.5) --
        (3.75, 0.5) -- (3.75,-1.75);
    \draw[join=round, line width = 5pt, color = gray!25] (1.75,-1.75) --
        (1.75,0.5) -- (3.75, 0.5) -- (3.75,-1.75);
    \draw (2.75,0.5) node {\tiny{KNO$_{3}$}};

    %Draw front of vessel 1

    \draw (0,0) .. controls +(-90:0.5) and +(-90:0.5) .. (2,0);
    \draw (0,0) .. controls +(-90:0.5) and +(-90:0.5) .. (2,0)
        -- (2,-0.5) .. controls +(-90:0.5) and +(-90:0.5) .. (0,-0.5) -- (0,0);

    %Second part

    \draw[fill=blue!60, fill opacity=0.5] (0,-0.5) .. controls +(-90:0.5)
    and +(-90:0.5) .. (2,-0.5);
    \draw[fill=blue!60, fill opacity=0.5] (0,-0.5) .. controls +(-90:0.5)
    and +(-90:0.5) .. (2,-0.5)
        -- (2,-2) .. controls +(-90:0.5) and +(-90:0.5) .. (0,-2) -- (0,-0.5);

    % draw voltmeter

    \draw[join = round, thick] (1,2) -- (1,2.5) -- (4.5,2.5) -- (4.5,2);
    \draw (2.75,2.5) node [circle, draw, fill=red!30] {V};

    %Draw back of vessel 2

    %Draw electrode

    \draw[fill=darkgray] (4,2) rectangle (5,-1);
    \draw (5.5,2.3) node {Zn};
    \draw (4.5,-1.75) node {\small{ZnSO$_{4}$}};

    % Draw front of vessel 2
    % part 1
    \draw (3.5,0) .. controls +(-90:0.5) and +(-90:0.5) .. (5.5,0);
    \draw (3.5,0) .. controls +(-90:0.5) and +(-90:0.5) .. (5.5,0)
        -- (5.5,-0.5) .. controls +(-90:0.5) and +(-90:0.5)
        .. (3.5,-0.5) -- (3.5,0);
    % part 2
    \draw[fill=lightgray, fill opacity=0.5] (3.5,-0.5) .. controls +(-90:0.5)
    and +(-90:0.5) .. (5.5,-0.5);
    \draw[fill=lightgray, fill opacity=0.5] (3.5,-0.5) .. controls +(-90:0.5)
        and +(-90:0.5) .. (5.5,-0.5) --
        (5.5,-2) .. controls +(-90:0.5) and +(-90:0.5)
        .. (3.5,-2) -- (3.5,-0.5);

\end{tikzpicture}


\end{document} 