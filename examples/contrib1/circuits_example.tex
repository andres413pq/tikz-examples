% Logic circuits library
% Authors: Juergen Werber and Christoph Bartoschek
\documentclass{article}

\usepackage{tikz}
% load circuits library
\usetikzlibrary{circuits}

%%%<
\usepackage{verbatim}
\usepackage[active,tightpage]{preview}
%\PreviewEnvironment{tikzpicture}
\setlength\PreviewBorder{5pt}%
%%%>
\begin{comment}
:Title: Logic circuits library
:Tags: Block diagrams, Nodes and shapes, Matrix

The ``circuits`` library, written by Juergen Werber and Cristoph Bartoschek, is a 
library of custom node shapes for drawing logic circuits. The library code is well
written and easy to follow. Adding new variants of the shapes is straightforward.

To run the example the ``pgflibrarytikzcircuits.code.tex`` file is needed:

- `pgflibrarytikzcircuits.code.tex`_

Note that the library is loaded using:

.. sourcecode:: latex

    \usetikzlibrary{circuits}

.. _pgflibrarytikzcircuits.code.tex: /media/pgftikzexamples/circuits/pgflibrarytikzcircuits.code.tex 

:Authors: Juergen Werber and Christoph Bartoschek


\end{comment}

\begin{document}

\begin{preview}

\begin{tikzpicture}[scale=0.5,cktbaselength=0.5pt]
\draw (6,0) node[nand2ni,anchor=tip,draw] (n4) {}
      (n4.z) -- +(0mm,-5mm)
      (n4.b) |- ++(10mm,2mm) -- ++(0mm,2mm)
      node[and2in,anchor=z,draw,fill=gray] (n3) {}
      (n4.a) |- ++(-10mm,2mm) -- ++(0mm,2mm)
      node[or3nni,anchor=tip,draw] (n2) {}
      (n3.b) -- +(0mm,5mm) node[above] {$i_5$}
      (n3.a) |- ++(-5mm,2mm)
      coordinate (p) -- +(0mm,5mm) node[above] {$i_4$}
      (n2.c) |- (p)
      (n2.b) |- ++(6mm,12mm) -- +(0mm,5mm) node[above] {$i_3$}
      (n2.a) -- +(0mm,8mm)
      node[and2,anchor=tip,draw] (n1) {}
      (n1.b) -- +(0mm,5mm) node[above] {$i_2$}
      (n1.a) -- +(0mm,5mm) node[above] {$i_1$};
\end{tikzpicture}
%
% Define a few helper macros to save typing
% The \spacer macro inserts a dummy node and defines the t (top) and
% b (bottom) coordinates. t and b are used to align the pins
\newcommand\spacer{%
\node[anchor=south,minimum height=1cm,inner sep=0pt, outer sep=0pt] (tmp) {};%
\path (tmp.north) coordinate (t) (tmp.south) coordinate (b);%
}
% Draw connecting pins. Note the use of the t and b coordinates defined
% in the \spacer macro
\newcommand\drawpins[1]{%
\draw (#1.a) -- (t -| #1.a)
      (#1.b) -- (t -| #1.b)
      (#1.z) -- (b -| #1.z);
}
% Draw pins for a logic gate with tree inputs
\newcommand\drawpinsiii[1]{%
\draw (#1.a) -- (t -| #1.a)
      (#1.b) -- (t -| #1.b)
      (#1.c) -- (t -| #1.c)
      (#1.z) -- (b -| #1.z);
}
%
\begin{tikzpicture}[scale=0.5,cktbaselength=0.5pt,baseline]
\tikzstyle{gate} = [draw,anchor=center,yshift=5mm]
\tikzstyle{ann} = [below,font=\small\tt]
\matrix[column sep=0.5cm] (circuit shapes) {
    % Use the first column to place a dummy spacer node
    \spacer &[-0.5cm] % remove column spacing
    \node[and2,gate] (N10) {} node[ann] {and2};\drawpins{N10} &
    \node[and2in,gate] (N20) {} node[ann] {and2in};\drawpins{N20} &
    \node[and2ni,gate] (N30) {} node[ann] {and2ni};\drawpins{N30} &
    \node[and2ii,gate] (N40) {} node[ann] {and2ii};\drawpins{N40} \\
    \spacer &
    \node[nand2,gate] (N11) {} node[ann] {nand2};\drawpins{N11} &
    \node[nand2in,gate] (N21) {} node[ann] {nand2in};\drawpins{N21} &
    \node[nand2ni,gate] (N31) {} node[ann] {nand2ni};\drawpins{N31} &
    \node[nand2ii,gate] (N41) {} node[ann] {nand2ii};\drawpins{N41} \\
    \spacer &
    \node[or2,gate] (N12) {} node[ann] {or2};\drawpins{N12} &
    \node[or2in,gate] (N22) {} node[ann] {or2in};\drawpins{N22} &
    \node[or2ni,gate] (N32) {} node[ann] {or2ni};\drawpins{N32} &
    \node[or2ii,gate] (N42) {} node[ann] {or2ii};\drawpins{N42} \\
    \spacer &
    \node[nor2,gate] (N13) {} node[ann] {nor2};\drawpins{N13} &
    \node[nor2in,gate] (N23) {} node[ann] {nor2in};\drawpins{N23} &
    \node[nor2ni,gate] (N33) {} node[ann] {nor2ni};\drawpins{N33} &
    \node[nor2ii,gate] (N43) {} node[ann] {nor2ii};\drawpins{N43} \\
    \spacer &
    \node[or3,gate] (N14) {} node[ann] {or3};\drawpinsiii{N14} &
    \node[nor3iii,gate] (N24) {} node[ann] {nor3iii};\drawpinsiii{N24} &
    \node[or3nni,gate] (N34) {} node[ann] {or3nni};\drawpinsiii{N34} \\
    \spacer &
    \node[buffer,gate] (buf) {} node[ann] {buffer};
    \draw (buf.z) -- (buf.z |- b)
          (buf.a) -- (buf.a |- t); &
    \node[inverter,gate] (inv) {} node[ann] {inverter};
    \draw (inv.z) -- (inv.z |- b)
          (inv.a) -- (inv.a |- t); \\
};%
\node[above] at (circuit shapes.north) {Circuit node shapes};%
\end{tikzpicture}%
\end{preview}

\end{document}
