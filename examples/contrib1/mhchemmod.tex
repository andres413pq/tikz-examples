% Oxidation and reduction
% Author: Arne Röhrs
% Note that two compilations are necessary. 
\documentclass{article}
\usepackage{tikz}
\usepackage[version=3,arrows=pgf]{mhchem}

%%%<
\usepackage{verbatim}
\usepackage[active,tightpage]{preview}
\setlength\PreviewBorder{5pt}%
%%%>

\begin{comment}
:Title: Oxidation and reduction
:Tags: Remember picture, Overlays

Example of how to add TikZ arrows to chemical reactions made with the `mhchem package`_.
TikZ coordinate nodes are put in the reaction equations and later used as reference points
for drawing using TikZ's ``remember picture`` feature. 


.. _mhchem package: http://www.ctan.org/tex-archive/help/Catalogue/entries/mhchem.html
\end{comment}

\begin{document}

% Insert a named coordinate for later reference.
% Usage:
% \ncoord{name}
% \ncoord[1em,2em]{name}
\newcommand\ncoord[2][0,0]{%
    \tikz[remember picture,overlay]{\path (#1) coordinate (#2);}%
}

% Define a handful of special cases of \ncoord
\newcommand\ccoord[1]{\ncoord[0.5em,0.9em]{#1}}%
\newcommand\bcoord[1]{\ncoord[0.4em,-0.3em]{#1}}%
\newcommand\dcoord[1]{\ncoord[0.9em,1.5em]{#1}}%
\newcommand\ecoord[1]{\ncoord[0.9em,-0.3em]{#1}}%

\tikzset{
    oxidation/.style={thick,red!80!black},
    reduction/.style={thick,blue!80!black},
}

\begin{preview}
~

% First reaction
\[
\ce{{\ccoord{a}} Na + {\bcoord{c}} Cl -> {\ccoord{b}} Na+  + {\bcoord{d}} Cl-}
\]
\tikz[overlay,remember picture] {
    \draw[oxidation,->] (a) -- ++(0,0.7em) -| (b)
        node[above, near start] {Oxidation};
    \draw[reduction,->] (c) -- ++(0,-0.7em) -| (d)
        node[below, near start] {Reduktion};
}
\\[0.5cm] % add some extra vertical spacing

% Second reaction
\[
    \ce{%
    {\dcoord{a}} $\overset{\text{0}}{\text{2\,Na}}$
    +{\ecoord{c}}$\overset{\text{0}}{\text{Cl$_2$}}$ ->
    {\dcoord{b}} $\overset{\text{+I}}{\text{2\,Na$^+$}}$  +
    {\ecoord{d}} $\overset{\text{-I}}{\text{2\,Cl$^-$}}$}
\]
\tikz[overlay,remember picture] {
    \draw[oxidation,->] (a) -- ++(0,0.7em) -| (b)
        node[above, near start] {Oxidation};
    \draw[reduction,->] (c) -- ++(0,-0.7em) -| (d)
        node[below, near start] {Reduktion};
}

\end{preview}

\end{document} 