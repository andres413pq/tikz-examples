% Scheme of Greatest Common Divisor
% Author: Giuseppe (joematara@hotmail.com)
\documentclass{article}
\usepackage{tikz}	
%%%<
\usepackage{verbatim}
\usepackage[active,tightpage]{preview}
\PreviewEnvironment{tikzpicture}
\setlength{\PreviewBorder}{10pt}%
%%%>
\begin{comment}
:Title: Scheme of Greatest Common Divisor (GCD)
:Tags: Mathematics
:Author: Giuseppe
:Slug: euclid-algorithm

Scheme of Greatest Common Divisor (GCD)
performed through the Euclidean Algorithm
and suitable to carry out manual calculations by
following the coloured arrows:

* inclined arrows stand for division operations
* horizontal ones for multiplication 
* vertical for subtraction 

Basically such items are the well known procedure of  
pupils at elementary schools. 

Remarks: code lines from 'Fill numbers' to the end  
         are written only for example purpose            
         and should be erased prior to print the scheme.
\end{comment}
\begin{document}
\begin{tikzpicture}[set style={{help lines}+=[dashed]}]
\draw[style=help lines] (0,0) grid +(12,12);
\draw                   (4,11) grid +(1,1);
\draw                   (6,10) grid +(1,1);
\draw                   (4,10) grid +(1,1); 
%-----
\draw                   (4,9) grid +(1,1);
\draw                   (6,9) grid +(1,1);
%-----
\draw                   (6,8) grid +(1,1);
\draw                   (4,8) grid +(1,1);
%
\draw                   (4,7) grid +(1,1);
\draw                   (6,7) grid +(1,1);
%
\draw                   (6,6) grid +(1,1);
\draw                   (4,6) grid +(1,1);
%
\draw                   (4,5) grid +(1,1);
\draw                   (6,5) grid +(1,1);
%
\draw                   (6,4) grid +(1,1);
\draw                   (4,4) grid +(1,1);
%
\draw                   (4,3) grid +(1,1);
\draw                   (6,3) grid +(1,1);

%------------------------------------------------------
% red1
\draw   [red,very thick,->]   (5,11.2) -- (6,10.8);
  \draw   [red,thick, ->]   (5.25,10.5) -- (5,10.5); 
  \draw   [red,thick, ->]   (6,10.5) -- (5.75,10.5);
\draw   (5.5,10.5) circle (0.2);
  \draw   [red,thick, ->]   (4.5,10.2) -- (4.5,9.8);
% blue1
\draw   [blue,very thick,->]   (6,10.2) -- (5,9.8);
  \draw   [blue,thick,->]   (5.75,9.5) -- (6,9.5); 
  \draw   [blue,thick,->]   (5,9.5) -- (5.25,9.5);
\draw   (5.5,9.5) circle (0.2); 
  \draw   [blue,thick, ->]   (6.5,9.2) -- (6.5,8.8);
%-----------------  
% red2
\draw   [red,very thick,->]   (5,9.2) -- (6,8.8);
  \draw   [red,thick, ->]   (5.25,8.5) -- (5,8.5); 
  \draw   [red,thick, ->]   (6,8.5) -- (5.75,8.5);
\draw   (5.5,8.5) circle (0.2);
  \draw   [red,thick, ->]   (4.5,8.2) -- (4.5,7.8);
% blue2
\draw   [blue,very thick,->]   (6,8.2) -- (5,7.8);
  \draw   [blue,thick,->]   (5.75,7.5) -- (6,7.5); 
  \draw   [blue,thick,->]   (5,7.5) -- (5.25,7.5);
\draw   (5.5,7.5) circle (0.2); 
  \draw   [blue,thick, ->]   (6.5,7.2) -- (6.5,6.8);
%-----------------  
% red3
\draw   [red,very thick,->]   (5,7.2) -- (6,6.8);
  \draw   [red,thick, ->]   (5.25,6.5) -- (5,6.5); 
  \draw   [red,thick, ->]   (6,6.5) -- (5.75,6.5);
\draw   (5.5,6.5) circle (0.2);
  \draw   [red,thick, ->]   (4.5,6.2) -- (4.5,5.8);

% -------- Fill numbers -----------
\node  at  (4.5,11.5) {2993};
\node  at  (6.5,10.5) {1095};
\node  at  (5.5,10.5) {2};  
\node  at  (4.5,10.5) {2190}; 
%---
\node  at  (6.5,9.5) {803}; 
\node  at  (5.5,9.5) {1}; 
\node  at  (4.5,9.5) {803};
%---
\node  at  (6.5,8.5) {292};
\node  at  (5.5,8.5) {2}; 
\node  at  (4.5,8.5) {584};
%
\node  at  (4.5,7.5) {219};
\node  at  (5.5,7.5) {1};
\node  at  (6.5,7.5) {219};
%---
\node  at  (6.5,6.5) {73};
\node  at  (5.5,6.5) {3};
\node  at  (4.5,6.5) {219};
%
\node  at  (4.5,5.5) {0};
\end{tikzpicture}
\end{document}