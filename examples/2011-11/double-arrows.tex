% Double Arrows a la Chef
% Author: Dominik Haumann
\documentclass{article}
\usepackage{tikz}
%%%<
\usepackage{verbatim}
\usepackage[active,tightpage]{preview}
\PreviewEnvironment{tikzpicture}
\setlength{\PreviewBorder}{10pt}%
%%%>
\begin{comment}
:Title: Double Arrows a la Chef
:Tags: Arrows;Decorations;Node Positioning;Block diagrams;Decorative drawings;Electrical engineering
:Author: Dominik Haumann
:Slug: double-arrows
This example shows how to use double arrows with a very specific
type of arrow head. Such double arrows are often used in control
theory to distinct vectors from scalars. The title includes the
addon "à la Chef" because Professors exist who want these arrows
to look exactly like this. :-)
\end{comment}
\usetikzlibrary{arrows, decorations.markings}

% for double arrows a la chef
% adapt line thickness and line width, if needed
\tikzstyle{vecArrow} = [thick, decoration={markings,mark=at position
   1 with {\arrow[semithick]{open triangle 60}}},
   double distance=1.4pt, shorten >= 5.5pt,
   preaction = {decorate},
   postaction = {draw,line width=1.4pt, white,shorten >= 4.5pt}]
\tikzstyle{innerWhite} = [semithick, white,line width=1.4pt, shorten >= 4.5pt]

\begin{document}

\begin{tikzpicture}[thick]
  \node[draw,rectangle] (a) {A};
  \node[inner sep=0,minimum size=0,right of=a] (k) {}; % invisible node
  \node[draw,rectangle,right of=k] (b) {B};
  \node[draw,rectangle,below of=a] (c) {C};

  % 1st pass: draw arrows
  \draw[vecArrow] (a) to (b);
  \draw[vecArrow] (k) |- (c);

  % 2nd pass: copy all from 1st pass, and replace vecArrow with innerWhite
  \draw[innerWhite] (a) to (b);
  \draw[innerWhite] (k) |- (c);

  % Note: If you have no branches, the 2nd pass is not needed
\end{tikzpicture}

\end{document}
