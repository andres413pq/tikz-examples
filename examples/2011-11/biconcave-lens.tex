\documentclass{article}
\usepackage{tikz}
%%%<
\usepackage{verbatim}
\usepackage[active,tightpage]{preview}
\PreviewEnvironment{tikzpicture}
\setlength{\PreviewBorder}{10pt}%
%%%>
\begin{comment}
:Title: Biconcave lens
:Tags: Intersections;Optics
:Author: Alain Matthes
:Slug: biconcave-lens

The example is made with the "intersections" library.
We fill directly the center part. We need to determine the angle defined
by the center I of an arc and a point of the arc (here D).

Originally posted to [TeX.SE](http://tex.stackexchange.com/a/62919/213).
\end{comment}
\usetikzlibrary{intersections}

\begin{document}
\begin{tikzpicture}
  \coordinate (I)  at (1.9,0);     
  \path [name path=arc1, draw=none](-1.9,-1.5) 
     arc[start angle=-90, end angle=90,radius=1.5];
  \path [name path=arc2, draw=none](1.9,1.5)   
     arc[start angle=90, end angle=270,radius=1.5];
  \path [name path=rect, draw=none](-0.9,-0.9) rectangle (0.9,0.9);

  \path [name intersections={of = arc1 and rect}];
  \coordinate (A)  at (intersection-1);
  \coordinate (B)  at (intersection-2);

  \path [name intersections={of = arc2 and rect}];
  \coordinate (C)  at (intersection-1);
  \coordinate (D)  at (intersection-2);

  \pgfmathanglebetweenpoints{\pgfpointanchor{I}{center}}{%
    \pgfpointanchor{D}{center}} 
  \let\tmpan\pgfmathresult 

  \fill[red!50] (A)--(D)  
        arc[start angle=\tmpan, end angle=360-\tmpan,radius=1.5] -- (B)
        arc[start angle=\tmpan-180, end angle=180-\tmpan,radius=1.5] ; 

  \draw [brown, ultra thick] (A) -- (D);
  \draw [brown, ultra thick] (B) -- (C) ;
  \draw [blue, thick] (-1.9,-1.5) arc[start angle=-90, end angle=90,radius=1.5];
  \draw [blue, thick] (1.9,1.5) arc[start angle=90, end angle=270,radius=1.5];

  \begin{scope}[>=latex]
    \draw [->] (-3,0) -- (3,0);
    \draw [->] (0,-1.25) -- (0,1.5);
  \end{scope}
\end{tikzpicture}
\end{document} 