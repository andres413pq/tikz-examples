% High harmonics
% Author: Fawad Karimi
\documentclass{article}
\usepackage{tikz}
%%%<
\usepackage{verbatim}
\usepackage[active,tightpage]{preview}
\PreviewEnvironment{tikzpicture}
\setlength{\PreviewBorder}{10pt}%
%%%>
\begin{comment}
:Title: High harmonics
:Tags: Fadings;Physics
:Author: Fawad Karimi
:Slug: high-harmonics
\end{comment}
\usetikzlibrary{fadings}
\renewcommand*{\familydefault}{\sfdefault}
\begin{document}
\begin{tikzpicture}
	%------------------fundamental and harmonics
   \fill[red,path fading=east] (0:0) -- (10:6) -- (-10:6) -- cycle;
   \fill[red,path fading=west] (0:0) -- (-10:-6) -- (10:-6) -- cycle;
   \fill[blue!70!,path fading=east] (0:0) -- (3:6) -- (-3:6) -- cycle;

    %-----------captions
    [above]{bla}
    \draw (2.5,0)node[right] {\small higher harmonics:xuv--soft x-ray } ;
  	\draw (-5.5,0.5)node[right] {\small pump laser} ;
  	\draw (-5.5,0)node[right]{\small fs pulsed laser:vis--nir};
    \node [below] at (0,-0.5) {gas cell filled with inert gas} ;
    %	\node [above] at (0, 0.5)
   %----------transparent gas cell
    \begin{scope}[fill opacity = 0.4]
    \draw[gray!25,fill=gray!10] (-1.5,0.5) rectangle +(3,-1);
  \end{scope} 
\end{tikzpicture}

\end{document}