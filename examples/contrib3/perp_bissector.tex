\documentclass{minimal}
\usepackage{tikz,xifthen}

\begin{document}
		
		\newcounter{index}
		\setcounter{index}{0}
		
		\begin{tikzpicture}[
				scale=1.0,
				MyPoints/.style={draw=blue,fill=white,thick},
				Segments/.style={draw=blue!50!red!70,thick},
				MyCircles/.style={green!50!blue!50,thin}
				]
			% Warning : all this is an artisanal way of computing points
			% on the perpendicular bissector of [AB]
			% It could very well be achieved with more powerfull tools...
			% (package tkz-2d, for example)
			\clip (-2.5,-2.5) rectangle (7,7.5);
			\draw[color=gray,step=1.0,dotted] (-2.1,-2.1) grid (6.1,7.1);
			\draw[->] (-2,0)--(6.5,0) node[right]{$x$};
			\draw[->] (0,-2)--(0,7) node[above]{$y$};
		
			% Feel free to change here coordinates of points A and B
			\pgfmathparse{-sqrt(2)}		\let\Xa\pgfmathresult
			\pgfmathparse{2}		\let\Ya\pgfmathresult
			\coordinate (A) at (\Xa,\Ya);
			\pgfmathparse{5}		\let\Xb\pgfmathresult
			\pgfmathparse{13/3}		\let\Yb\pgfmathresult			
			\coordinate (B) at (\Xb,\Yb);
			
			% Let I be the midpoint of [AB]
			\pgfmathparse{(\Xb+\Xa)/2} \let\XI\pgfmathresult
			\pgfmathparse{(\Yb+\Ya)/2} \let\YI\pgfmathresult
			\coordinate (I) at (\XI,\YI);	
					
			\draw[red,thick] (A)--(B);
			
			% deltaX and deltaY are coordinates of vector AB
			\pgfmathparse{\Yb-\Ya} \let\deltaY\pgfmathresult
			\pgfmathparse{\Xb-\Xa} \let\deltaX\pgfmathresult
			
			% NormeddeltaX and NormeddeltaY are the normalized values of these coordinates
			\pgfmathparse{sqrt(\deltaX*\deltaX+\deltaY*\deltaY)} \let\r\pgfmathresult
			\pgfmathparse{\deltaX/\r} \let\NormeddeltaX\pgfmathresult
			\pgfmathparse{\deltaY/\r} \let\NormeddeltaY\pgfmathresult			

			% R is a point on the perpendicular bissector of [AB],
			% far away from the midpoint...
			\pgfmathparse{\YI-10.0*\NormeddeltaX} \let\YR\pgfmathresult
			\pgfmathparse{\XI+10.0*\NormeddeltaY} \let\XR\pgfmathresult
			
			% S is the image of R by the symmetry of axis AB
			\pgfmathparse{2*\YI-\YR} \let\YS\pgfmathresult
			\pgfmathparse{2*\XI-\XR} \let\XS\pgfmathresult
			\coordinate (R) at (\XR,\YR);
			\coordinate (S) at (\XS,\YS);
			\draw (R)--(S);
			
			\foreach \i in {-3,-2,...,5}{
				\ifthenelse{\equal{\i}{0}}% Do not redraw the segment [AB]
					{}%
					{%
						\stepcounter{index}
						% P(i) is a variable point on the perpendicular bissector.
						% The distance between P(i) and P(i+1) is equal to 1
						\pgfmathparse{\YI-\i*\NormeddeltaX} \let\YP\pgfmathresult
						\pgfmathparse{\XI+\i*\NormeddeltaY} \let\XP\pgfmathresult
						\coordinate (P) at (\XP,\YP);
						\pgfmathparse{sqrt((\XP-\Xa)*(\XP-\Xa)+(\YP-\Ya)*(\YP-\Ya))}
						\let\radius\pgfmathresult
									
						\draw[MyCircles] (P) circle ({\radius});
						\draw[Segments] (P)--(A);
						\draw[Segments] (P)--(B);
						\fill[MyPoints] (P) circle (0.8mm) node[right]{$P_{\theindex}$};
					}%
				};
				
			\fill[MyPoints] (A) circle (0.8mm) node[left]{$A$};
			\fill[MyPoints] (B) circle (0.8mm) node[right]{$B$};
			\fill[MyPoints] (I) circle (0.8mm) node[right]{$I$};			
		\end{tikzpicture}
		
\end{document}
