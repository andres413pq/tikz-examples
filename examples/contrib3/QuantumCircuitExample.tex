% Quantum circuit
% Author: Mark S. Everitt

%%%<
% This code produces an image of a `quantum circuit' that produces
% a Greenberger-Horne-Zeilinger (GHZ) state, which is important
% for tests of nonlocality. This is an example of how simple it is
% to draw quantum circuits using TikZ.
%%%>

\documentclass[10pt]{article}

\usepackage[hang,small,bf]{caption}    % fancy captions

\usepackage{tikz}	
\usetikzlibrary{backgrounds,fit,decorations.pathreplacing}  % TikZ libraries

\newcommand{\ket}[1]{\ensuremath{\left|#1\right\rangle}} % Dirac Kets

%%%<
\usepackage{verbatim}
\usepackage[active,tightpage]{preview}
\PreviewEnvironment{tikzpicture}
\setlength\PreviewBorder{5pt}%
%%%>

\begin{comment}
:Title: Quantum circuit

This code produces an image of a "quantum circuit"  that produces
a Greenberger-Horne-Zeilinger (GHZ) state, which is important
for tests of nonlocality. This is an example of how simple it is
to draw quantum circuits using TikZ.

\end{comment}

\begin{document}

\begin{figure}
  \centerline{
    \begin{tikzpicture}[thick]
    %
    % `operator' will only be used by Hadamard (H) gates here.
    % `phase' is used for controlled phase gates (dots).
    % `surround' is used for the background box.
    \tikzstyle{operator} = [draw,fill=white,minimum size=1.5em] 
    \tikzstyle{phase} = [fill,shape=circle,minimum size=5pt,inner sep=0pt]
    \tikzstyle{surround} = [fill=blue!10,thick,draw=black,rounded corners=2mm]
    %
    % Qubits
    \node at (0,0) (q1) {\ket{0}};
    \node at (0,-1) (q2) {\ket{0}};
    \node at (0,-2) (q3) {\ket{0}};
    %
    % Column 1
    \node[operator] (op11) at (1,0) {H} edge [-] (q1);
    \node[operator] (op21) at (1,-1) {H} edge [-] (q2);
    \node[operator] (op31) at (1,-2) {H} edge [-] (q3);
    %
    % Column 3
    \node[phase] (phase11) at (2,0) {} edge [-] (op11);
    \node[phase] (phase12) at (2,-1) {} edge [-] (op21);
    \draw[-] (phase11) -- (phase12);
    %
    % Column 4
    \node[phase] (phase21) at (3,0) {} edge [-] (phase11);
    \node[phase] (phase23) at (3,-2) {} edge [-] (op31);
    \draw[-] (phase21) -- (phase23);
    %
    % Column 5
    \node[operator] (op24) at (4,-1) {H} edge [-] (phase12);
    \node[operator] (op34) at (4,-2) {H} edge [-] (phase23);
    %
    % Column 6
    \node (end1) at (5,0) {} edge [-] (phase21);
    \node (end2) at (5,-1) {} edge [-] (op24);
    \node (end3) at (5,-2) {} edge [-] (op34);
    %
    % Bracket
    \draw[decorate,decoration={brace},thick] (5,0.2) to
	node[midway,right] (bracket) {$\frac{\ket{000}+\ket{111}}{\sqrt{2}}$}
	(5,-2.2);
    %
    % Background Box
    \begin{pgfonlayer}{background} 
    \node[surround] (background) [fit = (q1) (op31) (bracket)] {};
    \end{pgfonlayer}
    %
    \end{tikzpicture}
  }
  \caption{
    A quantum circuit for producing a GHZ state using
    Hadamard gates and controlled phase gates.
  }
\end{figure}
\end{document}